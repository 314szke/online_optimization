\documentclass[11pt,a4paper]{article}
%\usepackage[pdftex]{graphicx,color}
\usepackage{graphicx,color}
\usepackage{amsmath,amssymb,dsfont,fullpage,epsfig,multirow,longtable}
\usepackage{epsfig,epstopdf,hhline}
\usepackage{cases}
\usepackage{wrapfig}
\usepackage{algorithm}
%\usepackage[noend]{algorithmic}
\usepackage{algorithmic}
\renewcommand{\algorithmicrequire}{\textbf{Input:}}
\renewcommand{\algorithmicensure}{\textbf{Output:}}
\usepackage[numbers]{natbib}
\usepackage[top=1 in, bottom=1 in, left=1 in, right=1 in, letterpaper]{geometry}
\usepackage{tikz}
\usetikzlibrary{arrows,positioning,decorations.pathreplacing,shapes}

\usepackage{thmtools}
\usepackage{thm-restate}
\usepackage{enumerate}

% Ours
\usepackage{enumerate}
\usepackage{mdframed}
\usepackage{setspace}
\setlength{\textfloatsep}{8pt}

%\declaretheorem[name=Theorem]{thm}

% for affiliation
\usepackage{authblk}

%%%%%%%%%% Start TeXmacs macrosƒ
\newenvironment{proof}{\noindent\emph{Proof\ }}{\hspace*{\fill}$\Box$\medskip}
\newenvironment{claimproof}{\noindent\emph{Proof of claim\ }}{\hspace*{\fill}$\Box$\medskip}
\newenvironment{plainproof}{\noindent\emph{Proof\ }}{}
\newtheorem{theorem}{Theorem}
\newtheorem{definition}{Definition}
\newtheorem{lemma}{Lemma}
\newtheorem{claim}{Claim}
\newtheorem{proposition}{Proposition}
\newtheorem{corollary}{Corollary}
\usepackage{amsmath}
\usepackage{paralist}
\usepackage{framed}

%%comment in algorithms
\renewcommand{\algorithmiccomment}[1]{\hfill  {\small  \tt \# #1}}

\newcommand\restr[2]{{% we make the whole thing an ordinary symbol
  \left.\kern-\nulldelimiterspace % automatically resize the bar with \right
  #1 % the function
  \vphantom{\big|} % pretend it's a little taller at normal size
  \right|_{#2} % this is the delimiter
  }}

\newcommand{\pred}{\texttt{pred}}
\newcommand{\vect}[1]{\ensuremath{\mathbf{#1}}}
\newcommand{\one}{\ensuremath{\mathds{1}}}
\newcommand{\PoA}{\text{PoA}}
\newcommand{\E}{\ensuremath{\mathbb{E}}}
% bullet dot
\newcommand\sbullet[1][.75]{\mathbin{\vcenter{\hbox{\scalebox{#1}{$\bullet$}}}}}

\usepackage[utf8]{inputenc} % Required for inputting international characters
\usepackage[T5]{fontenc} % Output font encoding for international characters

\usepackage{color, colortbl}
\usepackage{hyperref}
\hypersetup{colorlinks,
            linkcolor=blue,
            citecolor=blue,
            urlcolor=magenta,
            linktocpage,
            plainpages=false}
\usepackage[capitalise,noabbrev]{cleveref}



\newcommand{\comment}[1]{\textcolor{blue}{{\footnotesize
#1}}\marginpar{\raggedright\tiny \textcolor{blue}{Comment}}}

\begin{document}

\title{Online Covering with Multiple Experts}

\author{Kevi Enik\H{o}}
\author{Nguyễn Kim Thắng}
\affil{LIG, University Grenoble-Alpes, France }

\maketitle

\begin{abstract}
  Designing online algorithms with machine learning predictions is a recent technique beyond the worst-case paradigm for various practically relevant online problems (scheduling, caching, clustering, ski rental, etc.). While most previous learning-augmented algorithm approaches focus on integrating the predictions of a single oracle,
  we study the design of online algorithms with \emph{multiple} experts. To go beyond the popular benchmark of a static best expert in hindsight, we propose a new \emph{dynamic} benchmark (linear combinations of predictions that change over time).
  We present a competitive algorithm in the new dynamic benchmark with a performance guarantee of $O(\log K)$, where $K$ is the number of experts,
  for $0-1$ online optimization problems. Furthermore, our multiple-expert approach provides a new perspective on how to combine in an online manner several online algorithms - a long-standing central subject in the online algorithm research community.
\end{abstract}

%!TEX root = ./main.tex

\section{Introduction}

% Introduction and motivation for our problem
% main objective: compare the best combination of experts.
% So for in the literature, always compare to the best expert (ex., regret)

The domain of algorithms with predictions \cite{MitzenmacherVassilvitskii20:Beyond-the-Worst-Case}  (or learning-augmented algorithms) emerged recently and grew immensely at the intersection of (discrete) algorithm design and machine learning (ML).
Combining ML techniques with traditional algorithm design methods enables online algorithms to benefit from predictions that can infer future information from patterns in past data. Online algorithms with predictions can obtain performance guarantees beyond the worst-case analysis and provide fine-tuned solutions to various problems. In the literature, many significant problems have new learning-augmented results, for example, scheduling \cite{LattanziLavastida20:Online-scheduling,Mitzenmacher20:Scheduling-with}, paging \cite{LykourisVassilvtiskii18:Competitive-caching,Rohatgi20:Near-optimal-bounds,AntoniadisCoester20:Online-metric}, ski rental \cite{GollapudiPanigrahi19:Online-algorithms,KumarPurohit18:Improving-online,AngelopoulosDurr20:Online-Computation}, counting sketches \cite{HsuIndyk19:Learning-Based-Frequency}, bloom filters \cite{KraskaBeutel18:The-case-for-learned,Mitzenmacher18:A-model-for-learned}, and metrical task systems \cite{AntoniosEtAll23:mixing-predictions-metric-algorithms}.

Even though predictions provide a glimpse of the future, there is no mathematical guarantee for their accuracy. Adjusting the algorithm's trust in the predictions is a significant challenge since online algorithms must make irrevocable decisions at each time step. Ideally, if the predictions are accurate, the algorithm should perform well compared to the offline setting. In contrast, if the predictions are misleading, the algorithm should maintain a competitive solution, similar to the online setting where no predictive information is available. In other words, online algorithms with predictions are expected to bring the best of both worlds: mathematical performance guarantees of classical algorithms and good future prediction capabilities of machine learning methods.

Predictions can come from multiple sources (for example: heuristics, oracles, and randomized methods), but we ignore their nature and call all of them \emph{experts}.  An algorithm's consistency with the experts' suggestions is typically measured by comparing the algorithm's result with the solution of the \emph{best} expert. A representative example is the popular notion of regret in online learning, which fueled the development of many powerful algorithms and techniques.

A natural research question is whether it is possible to design competitive algorithms with mathematical performance guarantees with a stronger benchmark than the best expert. Comparing an algorithm with a stronger benchmark could provide deeper insights into the learning process and give better ways of exploiting the experts' predictions.

Taking a broader view, we can study whether combining predictions of several experts is similar to combining multiple online algorithms and whether we can expect to achieve better solutions with the combination. Assuming that we do not know in advance which of the given algorithms would perform best on the upcoming requests, can we combine the algorithms in some generic way to obtain a competitive online strategy? This has been a long-standing question in the community of online algorithms \cite{AzarBroder93:On-line-Choice,BlumBurch00:On-line-Learning}. To find an answer, it is a crucial to understand to what extent an online strategy can benefit from the input of multiple algorithms and which benchmark suits to evaluate its performance.

While in a completely general setting, such an online strategy and a corresponding benchmark may not exist, in this paper we propose
two algorithms for online linear and non-linear problems with covering constraints that are competitive with the new benchmark (the \emph{best linear combination} of the experts). Thus, our paper partially addresses the question we raised in the previous paragraph.

\subsection{Model and Problem}

\paragraph{Covering problem with experts.}
In the \emph{linear} problem setting, we have $n$ resources, and each resource $i$ has a cost per unit $c_{i}$ that we know in advance ($1 \leq i \leq n$).
Let $x_{i}$ be a non-negative variable representing the amount chosen from resource $i$.
The total cost of a solution $(x_{i})_{i=1}^{n}$ is $\sum_{i=1}^{n} c_{i} x_{i}$.
The problem includes $K$ experts, and the problem's covering-type constraints are revealed online one by one.
At each time $t \geq 1$, we receive a covering constraint $\sum_{i=1}^{n} a_{i}^{t} x_{i} \geq 1$ (where $a_{i}^{t} \geq 0$) and each expert $k$ (where $1 \leq k \leq K$) provides
a solution $(s_{i,k}^{t})_{i=1}^{n}$. Our algorithm can observe the experts' solutions, and afterwards, it must update its own solution (denoted as $(x_{i}^{t})_{i=1}^{n}$)
to satisfy the new constraint while maintaining the satisfaction of the previous ones. This algorithm must update its solution in the sense of online algorithms so it cannot modify the previously made decisions. Formally, $x_{i}^{t} \geq x_{i}^{t-1} ~\forall\ i, t$.
Our goal is to design an algorithm that minimizes $\sum_{i=1}^{n} c_{i} x_{i}^{T}$ subject to
all online covering constraints $t$, where $1 \leq t \leq T$. The value $T$ is the last time a constraint is released, and it is not known by the algorithm.

The \emph{non-linear} problem setting is analogous to the linear one, since we consider non-linear problems with linear constraints. The only difference compared to the previous paragraph is the total cost of the solution $(x_{i}^t)_{i=1}^{n}$, which becomes $f(x^t)$, where $f$ is a non-linear function. This setting captures different classes of problems: online mixed packing and covering, submodular optimization, etc.

\noindent \textbf{Experts.} \label{subsec:experts} In our model, the experts' predictions are also \emph{online solutions}. In other words, the experts' solutions
fulfill the following properties:
\begin{compactenum}
	\item for every expert $k$ and for every time $t$ the solution $(s_{i,k}^{t})_{i=1}^{n}$ is feasible, therefore, every constraint $t'$ where $1 \leq t' \leq t$ is satisfied;
	\item for every expert $k$ and for every time $t$ and for every resource $i$, the previous expert solutions are irrevocable, therefore $s_{i,k}^{t} \geq s_{i,k}^{t'}$ for all $t' \leq t$.
\end{compactenum}
These properties can be verified online. If some experts do not satisfy them, we simply ignore those experts both in the decision-making and in the benchmark.
A crucial remark: we do \emph{not} assume that the experts' solutions are tight at each constraint $t$, requiring $\sum_{i=1}^{n} a_{i}^{t} s_{i,k}^{t} = 1 ~ \forall t, k$ to hold.
This assumption is unrealistic and cannot be maintained in an online manner (see the discussion in Appendix~\ref{appix-tight-solutions}).
Besides, assuming tight constraint satisfaction would simplify the problem, while intuitively,
the difficulty of designing competitive algorithms comes from the lack of obvious ways to distinguish
good expert solutions from (probably many) non-efficient/misleading ones.

\noindent \textbf{Benchmark.}
We consider a dynamic benchmark that captures the \emph{best linear combination} of all experts' solutions \emph{over time}.
Informally, at any online time step, the benchmark can take a linear combination of the experts' solutions.
The linear combination can be \emph{dynamically} changed over time, and it can be different from previous combinations.
However, the benchmark's decisions are also online, so it cannot decrease the value of the decision variables ($x_{i}$).
We refer to our benchmark with the name \texttt{LIN-COMB} from now on.

%\begin{wrapfigure}{r}{0.7\textwidth}
\begin{figure}
	\vspace{-0.5cm}
	\begin{mdframed}
	\vspace{-0.2cm}
		\begin{align*}
			\text{Linear:}  \qquad \min \sum_{i=1}^{n} c_{i} x_{i}^{T} &= \sum_{i=1}^{n} c_{i} \sum_{t=1}^{T}\bigl( x_{i}^{t} - x_{i}^{t-1}\bigr) & \\
			\text{Non-linear:} \qquad \min f(x^{T}) \hspace{0.37cm} & & \\
			\text{s.t.} \hspace{1.63cm}
			\sum_{k=1}^{K} w_{k}^{t} &= 1 \qquad &\forall t \\
			%
			\qquad x_{i}^{t} &\geq \sum_{k=1}^{K} w_{k}^{t} s_{i,k}^{t}  \qquad &\forall i, t\\
			%
			\qquad x_{i}^{t} &\geq x_{i}^{t-1} \qquad &\forall i, t \\
			%
			\qquad w_{k}^{t} &\geq 0 \qquad &\forall t, k
		\end{align*}
		%where $1 \leq t \leq T$, $1 \leq i \leq n$, and $f$ is non-linear.
	\end{mdframed}
	\vspace{-0.2cm}
	\caption{\texttt{LIN-COMB} benchmark}
	\label{fig:benchmark}
	\vspace{-0.5cm}
\end{figure}
%\end{wrapfigure}

The \texttt{LIN-COMB} benchmark's formal description is visible on \cref{fig:benchmark}.
Let $w_{k}^{t} \geq 0$ be the weight assigned by the \texttt{LIN-COMB} benchmark to expert $k$ (where $1 \leq k \leq K$) at time
$1 \leq t \leq T$.
Since we consider a linear combination, the constraint $ \sum_{k=1}^{K} w_{k}^{t} = 1$ must hold.
%In the linear program, we consider the relaxed version of this constraint, where $\sum_{k=1}^{K} w_{k}^{t} \geq 1$.
The solution of \texttt{LIN-COMB} at time $t$ is ideally $x_{i}^{t} = \sum_{k=1}^{K} w_{k}^{t} s_{i,k}^{t}$,
however, $x_{i}^{t}$ must be larger than $x_{i}^{t-1}$.
Therefore, we set $x_{i}^{t} = \max\bigl\{\sum_{k=1}^{K} w_{k}^{t} s_{i,k}^{t},\ x_{i}^{t-1}\bigr\}$ for $1 \leq i \leq n$.
%


Since every expert's solution is feasible by our assumptions, at each time $t$ and for all resource $i$ (where $1 \leq i \leq n$),
the constructed solution $x_{i}^{t} \geq \sum_{k=1}^{K} w_{k}^{t} s_{i,k}^{t}$ constitutes a feasible solution to the covering constraints of the original covering problem.

\noindent Formally, for every constraint $t'$ with $t' \leq t$,
%
\begin{align*}
	\sum_{i=1}^{n} a_{i}^{t'} x_{i}^{t} &\geq
	%
	\sum_{i=1}^{n} a_{i}^{t'} \biggl( \sum_{k=1}^{K} w_{k}^{t} s_{i,k}^{t} \biggr)
	%
	= \sum_{k=1}^{K} w_{k}^{t}  \biggl( \sum_{i=1}^{n} a_{i}^{t'} s_{i,k}^{t} \biggr)
	%
	 \geq \sum_{k=1}^{K} w_{k}^{t} \geq 1
\end{align*}
%
where the second inequality holds due to the feasibility of the experts' solutions.
%

We highlight that the best-expert benchmark is included in \texttt{LIN-COMB}. We get the solution of this benchmark by setting $w^{t}_{k^{*}} = 1$ for all $t$, where $1 \leq t \leq T$, and $w^{t}_{k} = 0$ for all $k \neq k^{*}$,
where $k^{*}$ is the best expert (so $x_{i}^{t} = s_{i,k^{*}}^{t}$ for all $i$ and $t$).

\texttt{LIN-COMB} is strictly stronger than the best-expert benchmark for a general setting. We provide here an example.
In the makespan minimization problem, one assigns $n$ unit jobs to $n$ identical machines to minimize the maximum machine load.
There are $n$ experts and each expert $i$ assigns all jobs to machine $i$ for $1 \leq i \leq n$. Hence, the best expert has the maximum machine load of $n$ (the same for every expert). However, the optimal solution in \texttt{LIN-COMB} can choose $w_{i} = 1/n$ which results in the makespan of $1$. The solution corresponds to the assignment of one job to one machine.

\subsection{Our approach and contribution}

\paragraph{Approach.} We use the primal-dual approach to design competitive algorithms with the new \texttt{LIN-COMB} benchmark. First, we relax the linear program formulation of \texttt{LIN-COMB}, which serves as a lower bound. Then, we take the dual of the relaxation, which is a lower bound on the relaxation. Following the chain of lower bounds, the dual problem is a lower bound on the \texttt{LIN-COMB} benchmark.
%The formulations of the relaxation and its dual are detailed in \cref{sec:covering} for linear problems and in \cref{sec:convex} for non-linear problems. \textbf{\color{red} to change here }

Both of our proposed algorithms set the decision variables at every time step based on the solution of an internal program. Our approach is inspired by the
convex regularization method of \cite{BuchbinderChen14:Competitive-Analysis}.
When the objective cost is a linear function, it is well-known that the regularization function is a shifted entropy function.
These functions have been widely used, in particular in the recent breakthrough related to $k$-server \cite{BubeckCohen18:K-server-via-multiscale,BuchbinderGupta19:k-servers-with}
and metrical task system problems \cite{BubeckCohen21:Metrical-task},
in which the entropy functions are shifted by constant parameters.

A novel point in our approach is that the entropy function is shifted by the average of the experts' solutions.
Moreover, regarding the constraints of the internal program, instead of using the experts' solutions directly,
we define auxiliary solutions that guarantee tight constraint satisfaction.
Intuitively, this step is useful to avoid misleading experts' suggestions.

It is more challenging to find a regularization function for non-linear objective functions.
In our approach, we propose a new regularizer by coupling the entropy-like function with the gradient-Lipschitz property of the original covering problem's objective function.
This new regularization function allows us to analyze the non-linear objective setting similarly to the linear case.

\paragraph{Results.} Let $\rho$ be the maximum ratio between the experts' solutions on the resources.
Informally, $\rho$ represents the discrepancy across the experts' predictions.
Formally,
%We define the following parameter to establish the competitive ratio of our algorithm.
\[
	\rho := \max_{i} \max_{t',t''} \biggl\{\frac{\sum_{k=1}^{K} s_{i,k}^{t'}}{\sum_{k=1}^{K} s_{i,k}^{t''}} \biggr\}  \textnormal{ s.t. } \sum_{k=1}^{K} s_{i,k}^{t''} > 0.
\]

Our main results are the following.
First, we present an algorithm for the online linear covering problem that achieves an objective cost at most $O(\ln(K\rho))$ times the cost of the \texttt{LIN-COMB} benchmark.
For the online non-linear covering problem, we provide an algorithm that achieves an objective cost at most $O(\ln(K\rho)) \cdot \frac{\lambda}{(1-\mu\ln(K\rho))}$ times the cost of \texttt{LIN-COMB}, where ($\lambda$,$\mu$) are parameters of the non-linear objective function (precisely, ($\lambda$,$\mu$) are smoothness parameters of the objective function that we define later in the corresponding section).

In particular, for $0$-$1$ optimization problems, where the experts provide integer (deterministic or randomized) solutions, our first algorithm is $O(\ln(K))$-competitive with \texttt{LIN-COMB}, and the second one is $O(\ln(K)) \cdot \frac{\lambda}{(1-\mu\ln(K))}$-competitive.
Remarkably, our algorithms are resilient against the prediction quality fluctuations (discussed in \cref{subsec:related-works} and illustrated with experiments in \cref{sec:exp}).


\subsection{Related work and discussions} \label{subsec:related-works}

Much of the research focusing on surpassing worst-case performance guarantees is motivated by the spectacular advances of machine learning (ML). Specifically, ML methods can detect patterns among the arriving input requests and provide valuable insights for the online algorithms regarding future requests. \cite{LykourisVassilvtiskii18:Competitive-caching} introduced a general framework to integrate ML predictions into classical algorithm designs to surpass the worst-case performance limit.
As a result, many practically relevant online problems were revisited to enhance existing classical algorithms with ML predictions (see the aforementioned \cite{LattanziLavastida20:Online-scheduling,Mitzenmacher20:Scheduling-with,LykourisVassilvtiskii18:Competitive-caching,Rohatgi20:Near-optimal-bounds,AntoniadisCoester20:Online-metric,GollapudiPanigrahi19:Online-algorithms,KumarPurohit18:Improving-online,AngelopoulosDurr20:Online-Computation,HsuIndyk19:Learning-Based-Frequency,KraskaBeutel18:The-case-for-learned,Mitzenmacher18:A-model-for-learned,AntoniosEtAll23:mixing-predictions-metric-algorithms}).

On a high-level view, we aim to design algorithms that are robust (competitive) to the offline optimal solution and also consistent with the expert's predictions. Ideally, the performance of the designed algorithm should surpass previous bounds whenever the predictions are reliable (low errors).
However, most learning-augmented algorithms suffer when the error rates are neither very low nor very high, resulting in prediction confidence that is neither very low nor very high.
Figure~\ref{fig:robustness-consistency} provides a general picture of the performance of an algorithm with predictions, which is representative for many problems (for example, \cite{BamasMaggoriSvensson20:primal-dual-method,KeviNguyen23:Primal-Dual-Algorithms}).

\begin{wrapfigure}{r}{0.38\textwidth}
	\vspace{-0.4cm}
    \centering
    \includegraphics[width=0.4\textwidth]{./Img/consistency_robustness.png}
    \vspace{-0.8cm}
    \caption{Robustness-Consistency}
    \label{fig:robustness-consistency}
    \vspace{-0.3cm}
\end{wrapfigure}

In the figure, $\eta$ indicates the confidence in the predictions (or equivalently the error rate of predictions). The learning-augmented algorithm's performance bound is the maximum value of the green and orange curves (gray shaded area on the figure). We can observe that when $0.4 \leq \eta \leq 0.9$,
the algorithm's performance guarantee is worse than the classical worst-case guarantee (that can be achieved by simply ignoring all predictions).
Intuitively, in the case of neither very low nor very high confidence in the predictions, the algorithm has a hard time deciding if it should follow the predictions or the best-known standard algorithm in the worst-case paradigm.
It naturally raises the question of whether we can achieve at least a constant factor of the worst-case guarantee (where the constant is as close to 1 as possible) while assuring a resilient output solution regardless of the predictions' quality.
Our algorithms with the new benchmark provide an answer to this question.






%The paper of \cite{AnandGe22:Online-Algorithms} is the closest to ours, which also studies the design of algorithms with multiple experts.
%They consider a \texttt{DYNAMIC} benchmark that is intuitively
%the minimum cost solution that is supported by at least one expert solution at each time step. Formally:
%\[\texttt{DYNAMIC} = \min_{\hat{\textbf{x}} \in \hat{X}} \sum_{i=1}^{n} c_i \hat{x}_i \textnormal{, where}\]
%%
%$\hat{X} = \{\hat{\vect{x}} : \forall\ i \in [n],\ \forall\ t \in [T],\ \exists\ k \in [K]\ \textnormal{ s.t. } s_{i,k}^{t} \le \hat{x}_i \}$.
%%
%Our benchmark, \texttt{LIN-COMB}, is included in \texttt{DYNAMIC}, since every solution $x_{i}^{t}$ in \texttt{LIN-COMB} satisfies:
%$
%	x_{i}^{t} \geq \sum_{k=1}^{K} s_{i,k}^{t}w_{k}^{t} \geq \min_{k} \{s_{i,k}^{t}\}
%$.
%So for any $i$ and $t$, there exists $k$ such that $x_{i}^{t} \geq s_{i,k}^{t}$.
%However, the inverse is not true: a solution $\hat{\vect{x}}^{t} \in \hat{X}$ in \texttt{DYNAMIC} is not necessarily
%a linear combination of the experts' solutions.
%The \texttt{DYNAMIC} benchmark in \cite{AnandGe22:Online-Algorithms} relied on the assumption that at every time step
%the experts' solutions are tight. This assumption does not allow the representation of some realistic problems and it is impossible to maintain in online solutions (see Appendix~\ref{appix-tight-solutions}).
%Further, \cite{AnandGe22:Online-Algorithms} claimed an $O(\ln(K))$-competitive algorithm in the \texttt{DYNAMIC} benchmark.
%Unfortunately, their benchmark is too strong; we show an example in Appendix~\ref{sec:counter-example}
%in which their algorithm's performance guarantee is unbounded in their \texttt{DYNAMIC}
%benchmark. We believe that with a different benchmark their algorithm could be $O(\ln(K))$-competitive; however, we did not manage to prove this.


Integrating multiple predictions into online algorithms have been actively studied.
\cite{GollapudiPanigrahi19:skirental-multiple-predictions} studied the ski rental problem with multiple predictions.
The authors defined a consistency metric, which compares the performance of their algorithm to the optimal solution, given that at least one prediction (among the $k$ predictions) is optimal.
%By carefully integrating every prediction in their algorithm design, the authors managed to reduce the overall prediction error rate and obtain the best possible performance guarantee for their algorithm. During their analysis, they defined a consistency metric, which compares the performance of their algorithm to the optimal solution, given that at least one prediction (among the $k$ predictions) is correct.
\cite{AlmanzaChierichetti21:Online-Facility} also considered multiple predictions in the online facility location problem.
%The suggestions are treated as a family of sets and the authors use the union of these suggestions.
They compared the performance of their algorithm to the best possible solution obtained on the union of the suggestions. Recently, \cite{DinitzIm:Algorithms-with} studied the use of multiple predictors for several problems such as matching, load balancing, and non-clairvoyant scheduling. They provided algorithms that were competitive with the best predictor for such problems.
\cite{AnandGe22:Online-Algorithms} studies the design of algorithms with multiple experts for linear covering problems.
They consider a \texttt{DYNAMIC} benchmark that is intuitively
the minimum cost solution that is supported by at least one expert solution at each time step,
and provide an $O(\ln(K))$-competitive algorithm in the \texttt{DYNAMIC} benchmark.

%An important remark: all the above benchmarks are captured within \texttt{LIN-COMB}.

Furthermore, \cite{AntoniosEtAll23:mixing-predictions-metric-algorithms} proposed an algorithm with multiple experts for the metrical task system problem. Their benchmark allows switching from one expert to another at each time step, but it does not allow combinations of experts or any solution not suggested by one of the experts. In our \texttt{LIN-COMB} benchmark, the linear combination that evolves over time could result in a solution that is not suggested by any of the experts and potentially, it can be much more efficient. In \cite{AntoniosEtAll23:mixing-predictions-metric-algorithms} there is a cost for state transitions, which is appropriate in their setting, but in many other problems the smooth transition with additional costs from one state to another is not allowed (past decisions are immutable). Therefore, the results of \cite{AntoniosEtAll23:mixing-predictions-metric-algorithms} are not applicable to our setting.

Combining online algorithms into a new algorithm to achieve better results than the individual input algorithms has been a long-standing online algorithm design question \cite{AzarBroder93:On-line-Choice,BlumBurch00:On-line-Learning}.
Its intrinsic difficulty is similar to the issue we mentioned earlier: when the performance of the given input algorithms (or heuristics, randomized methods) is unclear (especially in the online setting), it is challenging to create a combination that can surpass the performance of the included algorithms.
Following the current development of online algorithm design techniques with multiple predictions, this subject has been renewed with different machine learning approaches. Our paper contributes to this line of research.

\subsection{Paper overview}
In this paper we show two algorithms to solve online \emph{linear} (\cref{sec:covering}) and online \emph{non-linear} (\cref{sec:convex}) covering problems with multiple experts.  \cref{sec:exp} shows empirical results, and we conclude in \cref{sec:conclusion}. The complete details can be found in the appendix.

%!TEX root = ./main.tex

\section{Online covering with multiple experts}	\label{sec:covering}

Our proposed algorithm solves online covering problems by creating linear combinations of the solutions proposed by $K$ experts in an online manner.
Recall that we evaluate the performance of our algorithm with the \texttt{LIN-COMB} benchmark (formalized on \cref{fig:benchmark}), which consists of the best linear combination of the experts' solution at each step.

Since our \texttt{LIN-COMB} benchmark is a linear combination of the experts' solutions, the equality $ \sum_{k=1}^{K} w_{k}^{t} = 1$ must hold, where $w_{k}^{t} \geq 0$ is the weight assigned to expert $k$ (where $1 \leq k \leq K$) at time $t$. In the following, we formulate a relaxed version of the \texttt{LIN-COMB} formulation, where
$\sum_{k=1}^{K} w_{k}^{t} \geq 1$. Additionally, the relaxed formulation enables us to avoid the (online) hard constraint requiring $w_{k}^{t} s_{i,k}^{t} \geq w_{k}^{t-1} s_{i,k}^{t-1}$ to hold, and instead, we introduce a new variable, $y_{i}^{t}$, to represent the increase of $x_{i}^{t}$ compared to $x_{i}^{t-1}$. When $w_{k}^{t} s_{i,k}^{t} < w_{k}^{t-1} s_{i,k}^{t-1}$ during the execution, we set the contribution of $i$ at time $t$ to be 0, and therefore, $y_{i}^{t} = 0$.

Due to the relaxed constraint, the optimal solution of the relaxed linear program is a lower bound of our \texttt{LIN-COMB} benchmark.
The relaxed formulation and its dual are displayed on \cref{fig:relaxation+dual}.

\begin{figure}[ht]
	\begin{mdframed}
		\begin{minipage}[t]{0.45\textwidth}
		\begin{align*}
			&& \min \sum_{t = 1}^{T} \sum_{i=1}^{n} & c_i y_i^t \\
			%
			(\alpha^{t})  && \sum_{k=1}^{K} w_{k}^{t} & \geq 1  & \forall\ t \\
			%
			(\beta_{i}^{t})  && \sum_{k=1}^{K} \left(w_{k}^{t} s_{i,k}^{t} - w_{k}^{t-1} s_{i,k}^{t-1} \right) &\leq y_i^t  &\forall\ i,t\\
		%
			&& w_{k}^{t},\ y_{i}^{t} & \ge 0 & \forall\ i,t,k
		\end{align*}
\end{minipage}
\quad
\begin{minipage}[t]{0.5\textwidth}
\begin{align*}
			 \max \sum_{t=1}^{T}  \alpha^{t} & \\
		%
			 \alpha^{t} + \sum_{i=1}^{n} s_{i,k}^{t} ( \beta_{i}^{t+1} - \beta_{i}^{t})   &\leq 0  &\forall\ k,t\\
		%
			 \beta_{i}^{t}   &\leq c_{i}  &\forall\ i,t \\
		%
			 \alpha_{i}^{t},\ \beta_{i}^{t} & \ge 0 & \forall\ i,t
		\end{align*}
\end{minipage}
	\end{mdframed}
	\caption{Relaxation of the \texttt{LIN-COMB} benchmark and its dual}
	\label{fig:relaxation+dual}
\end{figure}




%\begin{figure}[ht]
%	\begin{mdframed}
%		\begin{align*}
%			&& \min \sum_{t = 1}^{T} \sum_{i=1}^{n} & c_i y_i^t \\
%			%
%			(\alpha^{t}) \qquad && \sum_{k=1}^{K} w_{k}^{t} & \geq 1  & \forall\ t \\
%			%
%			(\beta_{i}^{t}) \qquad && \sum_{k=1}^{K} \left(w_{k}^{t} s_{i,k}^{t} - w_{k}^{t-1} s_{i,k}^{t-1} \right) &\leq y_i^t  &\forall\ i,t\\
%		%
%			&& w_{k}^{t},\ y_{i}^{t} & \ge 0 & \forall\ i,t,k
%		\end{align*}
%		%\vspace{5pt}
%	\end{mdframed}
%	\caption{Formulation of the relaxation of the \texttt{LIN-COMB} benchmark}
%	\label{fig:relaxation}
%\end{figure}
%
%\begin{figure}[ht]
%	\begin{mdframed}
%		\begin{align*}
%			&& \max \sum_{t=1}^{T} & \alpha^{t} \\
%		%
%			(x_{k}^{t}) \qquad && \alpha^{t} + \sum_{i=1}^{n} s_{i,k}^{t} ( \beta_{i}^{t+1} - \beta_{i}^{t})   &\leq 0  &\forall\ k,t\\
%		%
%			(y_{i}^{t}) \qquad && \beta_{i}^{t}   &\leq c_{i}  &\forall\ i,t \\
%		%
%			&& \alpha_{i}^{t},\ \beta_{i}^{t} & \ge 0 & \forall\ i,t
%		\end{align*}
%		%\vspace{5pt}
%	\end{mdframed}
%	\caption{Dual formulation of the relaxation of the \texttt{LIN-COMB} benchmark}
%	\label{fig:dual}
%\end{figure}


According to the theorem of weak duality, any feasible solution of the dual program lower bounds any feasible solution of the primal program, and therefore, any feasible dual solution also lower bounds our \texttt{LIN-COMB} benchmark. Following the chain of lower bounds, our approach to design a competitive algorithm is as follows. At every time step $t$, we build solutions for all $x_{i}^{t}$ together with the solutions for the dual problem $(\alpha^{t}, \beta_{i}^{t})$. Then, we bound the cost of the algorithm to that of the dual. It is important to emphasize that the designed solution for every $x_{i}^{t}$ must be feasible to the covering constraints, but it may \emph{not necessarily} be a linear combination of the experts' solutions.

\subsection{Competitive Algorithm} \label{sec:algo}

\paragraph{Preprocessing.}
Recall that by our assumptions, the experts' solutions are always feasible and non-decreasing. At the arrival of the $t^{\text{th}}$ constraint, expert $k$ (where $1 \leq k \leq K$) provides a feasible solution $s_{k}^{t} = (s_{i,k}^{t})_{i=1}^{n}$, such that $s_{i,k}^{t} \ge s_{i,k}^{t'}$ for all $t' \le t$ and all $i$ where $1 \le i \le n$. These assumptions do not exclude the possibility for the experts to provide malicious solutions that instruct the algorithm to use an unnecessarily large amount of resources.
Note that contrary to the assumption in \cite{AnandGe22:Online-Algorithms}, we can \emph{not} expect the experts' solutions to be always tight.
%Either the solutions come from an offline method and guarantee tightness as in \cite{BamasMaggoriSvensson20:primal-dual-method}, or the solutions are constructed online by the experts and do not guarantee tightness.
(In the supplementary file %Appendix~\ref{appix-tight-solutions}
we show an example that tight solutions cannot be maintained in an online manner.)


To circumvent this issue, we preprocess the experts' solutions at each iteration. During the preprocessing, every solution $s_k^t$ is scaled down to make it as tight as possible on the $t^{\text{th}}$ constraint, while always maintaining $s_{i,k}^{t} \geq s_{i,k}^{t-1}$ for all $i$. Additionally, after the down-scaling, we create an auxiliary solution $\hat{s}_k^t$ that is tight for the $t^{\text{th}}$ constraint. This solution is useful for our algorithm, and we create it with the following procedure.

After the down-scaling, do the following for each expert $k$.
\begin{compactenum}
	\item If $(s_{i,k}^{t})_{i=1}^{n}$ is tight on the $t^{\text{th}}$ constraint, then set $\hat{s}_{i,k}^{t} \gets s_{i,k}^{t}$  for every $i$.
	\item Let $\hat{s}_{i,k}^{t-1}$ be the auxiliary solution of expert $k$ at time $t-1$, meaning that, $\sum_{i=1}^{n} a_{i}^{t-1} \hat{s}_{i,k}^{t-1} = 1$. Given $I := \{i: s_{i,k}^{t} > \hat{s}_{i,k}^{t-1} \cdot \frac{a_{i}^{t-1}}{a_{i}^{t}} \}$, we set $\hat{s}_{i,k}^{t} \gets s_{i,k}^{t}$ if $i \notin I$
	and set $\hat{s}_{i,k}^{t}$ to be some value in $[\hat{s}_{i,k}^{t-1} \cdot \frac{a_{i}^{t-1}}{a_{i}^{t}}, s_{i,k}^{t}]$ if $i \in I$, s.t. the solution $\hat{s}_{i,k}^{t}$
	becomes tight on the $t^{\text{th}}$ constraint.
%	\item Otherwise, let $t' < t$ be the last constraint where expert $k$ provides a solution which is tight.
%	We define a set $I := \{i: s_{i,k}^{t} > s_{i,k}^{t'} \cdot \frac{a_{i}^{t'}}{a_{i}^{t}} \}$. Then, set $\hat{s}_{i,k}^{t} \gets s_{i,k}^{t}$ for $i \notin I$
%	and set $\hat{s}_{i,k}^{t}$ to be some value in $[s_{i,k}^{t'} \cdot \frac{a_{i}^{t'}}{a_{i}^{t}}, s_{i,k}^{t}]$ if $i \in I$, such that the solution $\hat{s}_{i,k}^{t}$
%	becomes tight on the $t^{\text{th}}$ constraint.
\end{compactenum}
%
\begin{lemma}
Following the preprocessing procedure, we can always obtain the solutions $\hat{s}_{i,k}^{t}$ such that
$\hat{s}_{i,k}^{t} \leq s_{i,k}^{t}$ and $\sum_{i=1}^{n} a_{i}^{t} \hat{s}_{i,k}^{t} = 1$. (Proof in the supplementary file.)
\end{lemma}
%%
%\begin{proof}
%Let us fix an expert $k$. We prove the lemma by induction. At time step $t=1$, one can always scale down the solution $s_{i,k}^{1} \geq 0$ such that the first constraint becomes tight.
%Assume that the lemma holds until $t-1$, $\sum_{i=1}^{n} a_{i}^{t-1} \hat{s}_{i,k}^{t-1} = 1$ and $\hat{s}_{i,k}^{t-1} \leq s_{i,k}^{t-1}$.
%Consider time $t$. If after scaling down (at the first step in the procedure) the $t^{\text{th}}$ constraint becomes tight, then we are done. Otherwise, we have
%	%
%	\begin{align*}
%		1 &< \sum_{i=1}^{n} a_{i}^{t} s_{i,k}^{t} = \sum_{i \in I} a_{i}^{t} s_{i,k}^{t} + \sum_{i \notin I} a_{i}^{t} s_{i,k}^{t}, \\
%		%
%		1 &= \sum_{i=1}^{n} a_{i}^{t-1} \hat{s}_{i,k}^{t-1} =  \sum_{i = 1}^{n} a_{i}^{t} \biggl( \hat{s}_{i,k}^{t-1} \cdot \frac{a_{i}^{t-1}}{a_{i}^{t}} \biggr) \\
%		&\geq  \sum_{i \in I} a_{i}^{t} \biggl( \hat{s}_{i,k}^{t-1} \cdot \frac{a_{i}^{t-1}}{a_{i}^{t}} \biggr)
%		+ \sum_{i \notin I} a_{i}^{t} s_{i,k}^{t}
%	\end{align*}
%	%
%	Hence, there exists $\hat{s}_{i,k}^{t} \in \bigl[ \hat{s}_{i,k}^{t-1} \cdot \frac{a_{i}^{t-1}}{a_{i}^{t}}, s_{i,k}^{t} \bigr]$ for every $i$, where $1 \leq i \leq n$, such that $\sum_{i=1}^{n} a_{i}^{t} \hat{s}_{i,k}^{t} = 1$.
%\end{proof}


\paragraph{Algorithm.}


At the arrival of the $t^{\text{th}}$ constraint,
\begin{compactenum}
	\item solve the following convex program and set $w^t$ to be the obtained optimal solution
%
\begin{align*}
&& \min_{w} \biggl\{\sum_{i=1}^{n} c_{i}  \biggl[  \biggl(\sum_{k=1}^{K} s_{i,k}^{t} w_{i,k}  + \delta_{i}^{t} \biggr) &
					 \ln \left( \frac{\sum_{k=1}^{K} s_{i,k}^{t} w_{i,k}  + \delta_{i}^{t}}{ \sum_{k=1}^{K}  s_{ik}^{t-1}w_{i,k}^{t-1}  + \delta_{i}^{t-1}}  \right)
					 		- \sum_{k=1}^{K}  s_{i,k}^{t} w_{i,k} \biggr] \biggr\} \\
%\end{align*}
%%
%\noindent subject to:
%%
%\begin{align*}
    (\gamma^{t})  && \sum_{i=1}^{n} a_{i}^{t} \biggl( \sum_{k=1}^{K}  \hat{s}_{i,k}^{t} w_{i,k} \biggr) &\geq 1 \qquad \forall\ t\\
%
    (\lambda_{i}^{t}) && \sum_{k=1}^{K}  w_{i,k} &\geq 1 \qquad \forall\ i\\
%
    (\mu_{i}^{t}) && \sum_{k=1}^{K} s_{i,k}^{t} w_{i,k} &\geq 0 \qquad \forall\ i,t
\end{align*}
%
where $\delta_{i}^{t} = \frac{1}{K} \sum_{k} s_{i,k}^{t}$.
%and recall that we set $\rho = \max_{i} \max_{t',t''} \left\{\frac{\sum_{k=1}^{K} s_{i,k}^{t'}}{\sum_{k=1}^{K} s_{i,k}^{t''}} : \sum_{k=1}^{K} s_{i,k}^{t''} > 0 \right\}$.
Note that in this program, we use the auxiliary solution $\hat{s}_{i,k}^{t}$ in the first constraint. For every $i$ where $s_{i,k}^{t} = 0$ for all $k$, the term related to $i$ is not included in the objective function of the convex program.
(We can set $w_{i,k} = 0$ for all $k$ beforehand.)
	%
	\item For all $i$ if $\sum_{k=1}^{K} w_{i,k}^{t} s_{i,k}^{t} > x_{i}^{t-1}$ then set $x_{i}^{t} \gets \sum_{k=1}^{K} w_{i,k}^{t} s_{i,k}^{t}$;
otherwise set $x_{i}^{t} \gets x_{i}^{t-1}$.
\end{compactenum}


\subsection{Analysis}
As $w^{t}$ is the optimal solution of the convex program and ($\gamma^t,\ \lambda_{i}^{t},\ \mu_{i}^{t}$) is the optimal solution of its dual, the following Karush-Kuhn-Tucker (KKT) and complementary slackness conditions hold.

\begin{align*}
   \biggl[ \sum_{i=1}^{n} a_{i}^{t} \biggl( \sum_{k}  \hat{s}_{i,k}^{t} w_{i,k}^{t} \biggr) - 1 \biggr] \gamma^{t} &= 0 \qquad \forall t \\
   \biggl[ \sum_{k=1}^{K}  w_{i,k}^{t}  - 1 \biggr] \lambda_{i}^{t} = 0 \qquad \forall i, t
   \qquad \qquad \qquad \qquad
   \biggl[ \sum_{k=1}^{K}  s_{i,k}^{t} w_{i,k}^{t} \biggr] \mu_{i}^{t} &= 0 \qquad \forall i, t \\
%
 c_{i} s_{ik}^{t} \ln \left( \frac{\sum_{k=1}^{K} s_{i,k}^{t} w_{i,k}^{t} + \delta_{i}^{t}}{\sum_{k=1}^{K}  s_{ik}^{t-1}w_{i,k}^{t-1}  + \delta_{i}^{t-1}} \right)
    	- a_{i}^{t} \hat{s}_{i,k}^{t} \gamma^{t} - \lambda_{i}^{t} - s_{i,k}^{t} \mu_{i}^{t} &= 0	\qquad \forall i,k,t \\
	%
	\gamma^{t}, \lambda_{i}^{t}, \mu_{i}^{t} &\geq 0 \qquad \forall i, t
\end{align*}

Moreover, if $\sum_{k=1}^{K} w_{i,k}^{t} s_{i,k}^{t} > 0$, meaning that $\mu_{i}^{t} = 0$, then
\begin{align}	\label{eq:KKT}
   c_{i} s_{ik}^{t} \ln \left( \frac{\sum_{k=1}^{K} s_{i,k}^{t} w_{i,k}^{t}  + \delta_{i}^{t}}{\sum_{k=1}^{K}  s_{ik}^{t-1}w_{i,k}^{t-1}  + \delta_{i}^{t-1}} \right)
    	- a_{i}^{t} \hat{s}_{i,k}^{t} \gamma^{t} - \lambda_{i}^{t} = 0
\end{align}


\paragraph{Dual variables and feasibility.} We set the dual variables of the linear program relaxation of our \texttt{LIN-COMB} benchmark based on the dual variables of the convex program used inside the algorithm.
%
\begin{align*}
    \alpha^{t} &= \frac{1}{\ln(K\rho)}  \biggl( \gamma^{t} + \sum_{i} \lambda_{i}^{t} \biggr), \qquad
    \beta_{i}^{t} = \frac{1}{\ln(K\rho)} c_i \ln \left(\frac{ (1 + 1/K) \cdot \max_{t'} \sum_{k=1}^{K} s_{i,k}^{t'}}{\sum_{k=1}^{K}  s_{i,k}^{t-1} w_{i,k}^{t-1} + \delta_{i}^{t-1}}\right)
\end{align*}
%
where recall that $\rho = \max_{i, t',t''} \left\{\frac{\sum_{k=1}^{K} s_{i,k}^{t'}}{\sum_{k=1}^{K} s_{i,k}^{t''}} : \sum_{k=1}^{K} s_{i,k}^{t''} > 0 \right\}$.

\begin{lemma} \label{lem:covering-feasibility}
The $x_{i}^{t}$ solutions set by the algorithm for the original covering problem and the dual variables $(\alpha^{t}, \beta_{i}^{t})$ of the \texttt{LIN-COMB} benchmark's linear program relaxation are feasible. (Proof in the supplementary file.)
\end{lemma}
%%
%\begin{proof}
%We first prove that the $x_{i}^{t}$ variables satisfy the covering constraints by induction. At time 0, no constraint has been released yet, and every variable is set to 0. This all-zero solution is feasible. Let us assume that the algorithm provides feasible solutions up to time $t-1$. At time $t$, the algorithm maintains the inequality $x_{i}^{t} \geq x_{i}^{t-1}$, so all constraints $t'$ where $t' < t$ are satisfied. Besides, $x_{i}^{t}$ is always at least
%$\sum_{k} w_{i,k}^{t} s_{i,k}^{t}$, which is larger than $\sum_{k} w_{i,k}^{t} \hat{s}_{i,k}^{t}$ since $s_{i,k}^{t} \geq \hat{s}_{i,k}^{t}$
%for all $i,k$ by the preprocessing step. Hence, the constraint $t$ is also satisfied, formally,
%$$
%\sum_{i=1}^{n} a_{i}^{t} x_{i}^{t}  \geq \sum_{i=1}^{n} a_{i}^{t} \biggl( \sum_{k} \hat{s}_{i,k}^{t} w_{i,k}^{t} \biggr) \geq 1.
%$$
%
%In the remaining part of the proof, we show the feasibility of $\alpha^{t}$ and every $\beta_{i}^{t}$.
%Since $ \gamma^{t} \geq 0$ and $\lambda_{i}^{t} \geq 0$ for all $i$ and $t$, we get that $\alpha^{t} \geq 0$.
%In the definition of  $\beta_{i}^{t}$, the nominator of the logarithm term is always larger than the denominator, and it is smaller than $K\rho$ times the denominator. Consequently, $0 \leq \beta_{i}^{t} \leq c_{i}$. Furthermore,
%%
%\begin{align*}
%    \beta_{i}^{t+1} - \beta_{i}^{t}
%    	&= - \frac{1}{\ln(K\rho)} c_i \ln \left( \frac{\sum_{k=1}^{K}  s_{i,k}^{t} w_{i,k}^{t} + \delta_{i}^{t}}{\sum_{k=1}^{K}  s_{i,k}^{t-1}w_{i,k}^{t-1} + \delta_{i}^{t-1}} \right).
%\end{align*}
%%
%Since $\sum_{i} a_{i}^{t} \hat{s}_{i,k}^{t} = 1$, using the KKT conditions, we get:
%\begin{align*}
%\alpha^{t} &+ \sum_{i=1}^{n} s_{ik}^{t} \left(\beta_{i}^{t+1} - \beta_{i}^{t}\right) \\
%%
%&= \frac{1}{\ln(K\rho)} \biggl( \gamma^{t} + \sum_{i} \lambda_{i}^{t} \biggr)
%	- \frac{1}{\ln(K\rho)}  \sum_{i=1}^{n} s_{i,k}^{t} c_i \ln \left( \frac{\sum_{k=1}^{K}  s_{ik}^{t} w_{i,k}^{t} + \delta_{i}^{t}}{\sum_{k=1}^{K}  s_{ik}^{t-1}w_{i,k}^{t-1} + \delta_{i}^{t-1}} \right) \\
%%
%&= \frac{1}{\ln(K\rho)} \biggl[ \gamma^{t} + \sum_{i=1}^{n} \lambda_{i}^{t} - \sum_{i=1}^{n} \left( a_{i}^{t} \hat{s}_{i,k}^{t} \gamma^{t} + \lambda_{i}^{t} + s_{i,k}^{t} \mu_{i}^{t} \right) \biggr] \\
%%
%&\leq 0
%\end{align*}
%\end{proof}


\begin{theorem} \label{covering-theorem}
The algorithm's cost is at most $O(\ln(K \rho))$-competitive in the \texttt{LIN-COMB} benchmark.
\end{theorem}
%
\begin{proof}
\cref{lem:covering-feasibility} proved that our algorithm creates feasible solutions for the dual problem of the \texttt{LIN-COMB} benchmark relaxation and for the original covering problem. We show that the algorithm's solution increases the primal objective value of the original covering problem by at most $O(\ln(K \rho))$ times the value of the dual solution, which serves as the lower bound on the \texttt{LIN-COMB} benchmark - the best linear combination of the experts' solutions.
\begin{align}
	 \sum_{i=1}^{n} &c_{i} (x_{i}^{t} - x_{i}^{t-1})
		= \frac{1}{\ln(K\rho)} \sum_{i: x_{i}^{t} > x_{i}^{t-1}} c_{i}(x_{i}^{t} - x_{i}^{t-1}) &&  \notag \\
		%
		&\leq \sum_{i: x_{i}^{t} > x_{i}^{t-1}} c_{i}(x_{i}^{t} + \delta_{i}^{t}) \ln \frac{x_{i}^{t-1} + \delta_{i}^{t}}{x_{i}^{t-1} + \delta_{i}^{t}} \\
		%
		&\leq \sum_{i: x_{i}^{t} > x_{i}^{t-1}} c_{i} (x_{i}^{t} + \delta_{i}^{t}) \ln \frac{x_{i}^{t-1} + \delta_{i}^{t}}{x_{i}^{t-1} + \delta_{i}^{t-1}} \\
		%
		&= \sum_{i: x_{i}^{t} > x_{i}^{t-1}} c_{i} \left[ \left(\sum_{k=1}^{K}  s_{i,k}^{t} w_{i,k}^{t} + \frac{1}{K} \sum_{k=1}^{K} s_{i,k}^{t} \right)
			\ln \left(\frac{ \sum_{k=1}^{K}  s_{i,k}^{t} w_{i,k}^{t} + \delta_{i}^{t}}{x_{i}^{t-1} + \delta_{i}^{t-1}}  \right) \right] \\
%
&\leq \sum_{i: x_{i}^{t} > x_{i}^{t-1}} c_{i} \left[ \left(\sum_{k=1}^{K}  s_{i,k}^{t} w_{i,k}^{t} + \frac{1}{K} \sum_{k=1}^{K} s_{i,k}^{t} \right)
			\ln \left(\frac{ \sum_{k=1}^{K}  s_{i,k}^{t} w_{i,k}^{t} + \delta_{i}^{t}}{\sum_{k=1}^{K}  s_{i,k}^{t-1} w_{i,k}^{t-1} + \delta_{i}^{t-1}}  \right) \right]\\
%
	&= \sum_{i: x_{i}^{t} > x_{i}^{t-1}} \sum_{k=1}^{K} (w_{i,k}^{t} + 1/K) c_{i} s_{i,k}^{t}
				\ln \left(\frac{ \sum_{k=1}^{K} s_{i,k}^{t} w_{i,k}^{t}  + \delta_{i}^{t}}{\sum_{k=1}^{K}  s_{i,k}^{t-1} w_{i,k}^{t-1}  + \delta_{i}^{t-1}}  \right) \notag \\
%
%\end{align}
%%
%\begin{align}
%
&=  \sum_{i: x_{i}^{t} > x_{i}^{t-1}} \sum_{k=1}^{K} (w_{i,k}^{t} + 1/K) \biggl( a_{i}^{t} \hat{s}_{i,k}^{t} \gamma^t + \lambda_{i}^{t} \biggr) \\
%
%
&\leq \sum_{i=1}^{n} \sum_{k=1}^{K} (w_{i,k}^{t} + 1/K) \biggl( a_{i}^{t} \hat{s}_{i,k}^{t} \gamma^t + \lambda_{i}^{t} \biggr) \notag \\
%
&= \sum_{i=1}^{n} a_{i}^{t} \biggl(\sum_{k=1}^{K} w_{i,k}^{t} \hat{s}_{i,k}^{t} \biggr) \gamma^t + \sum_{i=1}^{n} \bigg( \sum_{k=1}^{K} w_{i,k}^{t} \biggr) \lambda_{i}^{t}
+ \frac{1}{K}  \sum_{k=1}^{K} \biggl( \sum_{i=1}^{n} a_{i}^{t}  \hat{s}_{i,k}^{t}  \biggr) \gamma^t + \frac{1}{K} \sum_{k=1}^{K} \sum_{i=1}^{n} \lambda_{i}^{t} 		\notag \\
%
&= 2 \gamma^{t} + 2\sum_{i=1}^{n} \lambda_{i}^{t} = \ln(K \rho) \alpha^{t}
\end{align}
%
The above corresponding transformations hold since:
\begin{compactenum}[(1)]
	\setcounter{enumi}{1}
	\item follows from the inequality $a - b \leq a \ln(a/b)$ for all $0 < b \leq a$;
	\item holds since $\delta_{i}^{t} \geq \delta_{i}^{t-1}$ (because $s_{i,k}^{t} \geq s_{i,k}^{t-1}$ for all $i,k,t$);
	\item is valid because $x_{i}^{t} > x_{i}^{t-1}$, so $x_{i}^{t} = \sum_{k=1}^{K}  s_{i,k}^{t} w_{i,k}^{t}$;
	\item is by the design of the algorithm: $x_{i}^{t-1} \geq \sum_{k=1}^{K}  s_{i,k}^{t-1} w_{i,k}^{t-1}$;
	\setcounter{enumi}{5}
	\item since given that $x_{i}^{t} > x_{i}^{t-1} \geq 0$
	(so $\sum_{k=1}^{K}  s_{i,k}^{t} w_{i,k}^{t} = x_{i}^{t} > 0$), the KKT condition (\ref{eq:KKT}) applies;
	\item is true due to the complementary slackness conditions
		and that $\sum_{i=1}^{n} a_{i}^{t}  \hat{s}_{i,k}^{t} = 1$.
\end{compactenum}
\end{proof}

\begin{corollary} \label{corollary}
	For $0$-$1$ optimization problems in which experts provide integer (deterministic or randomized) solutions,
	the algorithm is $O(\ln K)$-competitive in the \texttt{LIN-COMB} benchmark.
	Subsequently, there exists an algorithm such that its performance is $O(\ln K)$-competitive in the \texttt{LIN-COMB}
	benchmark and is up to a constant factor to the best guarantee in the worst-case benchmark
\end{corollary}
%
\begin{proof}

\begin{wrapfigure}{r}{0.4\textwidth}
 \vspace{-0.9cm}
  \begin{center}
    \includegraphics[width=0.4\textwidth]{../paper/Img/algo_structure.pdf}
  \end{center}
  \vspace{-0.5cm}
      \caption{Structural overview of the algorithm's components. $E_1,\ E_2, \dots\ E_K$ correspond to the experts of the online problem.
      On the second layer, we integrate the best standard online algorithm with our algorithm. }
  \label{fig:algo-layers}
   \vspace{-0.5cm}
\end{wrapfigure}

	If the value of $s_{i,k}^{t}$ is in $\{0,1\}$ for every $i,k,t$, then
	\[
	\rho = \max_{i} \max_{t',t''} \left\{\frac{\sum_{k=1}^{K} s_{i,k}^{t'}}{\sum_{k=1}^{K} s_{i,k}^{t''}} : \sum_{k=1}^{K} s_{i,k}^{t''} > 0 \right\}
	\leq \frac{K}{1}
	\]
	Therefore, the competitive ratio of the main algorithm in the \texttt{LIN-COMB} benchmark is $O(\log K \rho) = O(\log K^2)$.

 	To obtain an algorithm that is competitive in both the \texttt{LIN-COMB} and the worst-case benchmarks, we proceed as follows (an illustration in \cref{fig:algo-layers}).
	We first apply the main algorithm on the $K$ experts' predictions to obtain an online algorithm, named $A$.
	Algorithm $A$ is $O(\ln K)$-competitive in the \texttt{LIN-COMB} benchmark. Let $B$ be the algorithm with the best worst-case guarantee.
	One applies the main algorithm one more on two algorithms, $A$ and $B$. The final algorithm is $O(\ln 2)$-competitive to both $A$ and $B$.
	In other words, its performance is $O(\ln K)$-competitive in the \texttt{LIN-COMB} benchmark and is up to a constant factor worse than the best guarantee in the worst-case benchmark.

\end{proof}

By \cref{corollary}, given a $0$-$1$ optimization problem, if there are $K$ deterministic online algorithms, then
we can design an algorithm that has a cost at most $O(\log K)$ times that of the best linear combination of those algorithms at any time.
Similarly, if $K$ given online algorithms are randomized (they output $0$-$1$ solutions with probabilities), then our algorithm
has an expected cost (randomization over the product of the distributions of those solutions) at most $O(\log K)$ times that of
the best linear combination of those algorithms at any time. Many practical problems admit $0$-$1$ solutions, for which our algorithm is of interest.
Consider problems like network design, ski rental, TCP acknowledgement, facility location, etc. Given the fractional solutions constructed by our algorithm,
we can apply existing online rounding schemes to obtain integral solutions for such problems.

%!TEX root = ./main.tex

\section{Experiments}

\paragraph{Setting.}
To evaluate the empirical performance of our proposed algorithm, we conducted experiments on routing problems
that are motivated by congestion management.
In the routing problem, we are given a directed graph $G(A,V)$, a set of requests $R = \{(s_{i}, t_{i}) : s_{i}, t_{i} \in V\}$ that represents demands of
connecting $s_{i}$ to $t_{i}$. We assume that for each request, there exists a directed path between $s_{i}$ to $t_{i}$.
Each arc $(u, v) \in A$ is associated with a cost function $f_{(u,v)}: \mathbb{R}^{+} \rightarrow \mathbb{R}^{+}$ that depends on the number of requests using the arc.
Requests arrive online, and one needs to design a routing that minimizes the total cost.


\paragraph{Input.}
We generate the input graphs randomly following the Erd\H{o}s-Rényi model $G(n, p)$, where $n$ is the number of vertices and $p$ is the probability that an arc gets created. The source and target vertices of the requests are also generated uniformly at random.

\paragraph{Predictions.}
The predictions rely on the optimal offline integral solution.
We define the error of a prediction $P$ on instance $I$ as $error(P(I)) = 1 - \frac{OPT(I)}{P(I)}$,
where $OPT(I)$ is the objective value of the optimal offline integral solution of instance $I$ and $P(I)$ is the objective value obtained by the prediction's solution.
To introduce errors in the prediction, we choose a request uniformly at random and attempt to find an alternative path compared to the optimal integral solution. We repeat this process several times to raise the error rate above the desired threshold.

\paragraph{Implementation.}
The covering formulation of the routing problem enumerates all possible cuts in the graph. At each arriving request $r = (s,t)$, our algorithm receives a set of constraints, $\sum_{e \in \delta(S)} x_{e}^{r} \ge 1$, where $\delta(S)$ is the cut on $S \subset V$, such that $s \in S$ and $t \notin S$.
%This formulation generates exponential number of constraints with respect to the size of the graph at each time step, which makes the implementation of the algorithm rather impractical.
%To circumvent this limitation, we slightly modify the implementation of our algorithm.
Upon each arriving request, our algorithm receives two solutions: one from the prediction and one from a greedy algorithm. The greedy algorithm calculates, at each arriving request, a path that minimizes the increase of the total cost and routes the request on this path.
It is shown that this routing has the optimal competitive ratio when the cost functions are polynomial \cite[Section 4.2]{Thang20:Online-Primal-Dual}.
In our implementation, we update the arcs %of both solutions with the same method as
as described in Algorithm~\ref{algo:covering}.
%The arcs that do not belong to either the greedy solution or the prediction are always dominated by the greedy solution.
The request is satisfied when a path exists among the arcs in the set $A^{*}$ (arcs with value $1$) in our algorithm. If such a path exists, the solution of the request is this path.
%, therefore the implementation includes a rounding step to obtain an integral solution.

\paragraph{Instances.} We show the results of several instances in Figure~\ref{fig:experiment} and other figures  in Appendix~\ref{appix:experiments}. The graph of the experiment contains $40$ vertices, $126$ arcs, and $17$ requests. The cost functions of the arcs are polynomials of degree 4, and the coefficients were generated randomly in $[1.0, 10.0]$.

\paragraph{Observation.} When the prediction is not an optimal offline solution (the error is not 0),
our algorithm outperforms both the prediction and the greedy solution when the confidence parameter is $0.1 \leq \eta \leq 0.8$.
In practice, predictions are neither perfect nor completely wrong. So the confidence parameter, reflecting the reliability of the predictions,
is rarely very close to 0 nor very close to 1. The experiments prove that our algorithm provides improvements over the predictions and the greedy solution
(achieving the best theoretical performance) in the practical aspect of the routing problem.

%because the algorithm looks for a solution using the union of the arcs belonging to the solutions of the prediction and the greedy algorithm.

\begin{figure}
    \includegraphics[width=\linewidth]{Img/figure1.pdf}
    \caption{The x-axis show the confidence in the prediction, where 0 means higher confidence. The y-axis show the competitive ratio compared to the optimal offline integral solution. The different colors (also markers) show the result of the algorithm with different prediction error rates and the solutions of the greedy algorithm and the prediction alone.}
    \label{fig:experiment}
\end{figure}

%!TEX root = ./main.tex

\section{Conclusion}

We introduce a dynamic \texttt{LIN-COMB} benchmark in the setting of multiple expert predictions beyond the traditional static benchmark of the best expert in hindsight
and give a competitive algorithm for the online covering problem in this benchmark.
Our approach can provide valuable insights into the learning processes related to predictions,
in particular, in aggregating information from predictions to improve the performance of existing algorithms,
and how to combine online algorithms, an important subject in the online algorithm design community \cite{AzarBroder93:On-line-Choice,BlumBurch00:On-line-Learning}.
The experiments support the fact that our algorithm can differentiate between good and adversarial experts to some extent.

An interesting open question is to design competitive algorithms in the \texttt{LIN-COMB} benchmark for different classes of problems,
such as packing problems and problems with non-linear objectives.

\bibliographystyle{plainurl}
\bibliography{references}

\clearpage

\appendix
%!TEX root = ./main.tex

\section{An example from the introduction} \label{apix:example-introduction}

The following example demonstrates why an approach where we alternate between the solutions of the prediction oracle and the primal-dual method does not work for non-linear objective functions.

Consider the objective of $x_{11}^p + ... + x_{1k}^p + x_{21}^p + ... + x_{2k}^p$ where $k$ and $p$ are some integers, and two algorithms: Algorithm $1$ and Algorithm $2$. At every time $t$, the constraint is $x_{1t} + x_{2t} \geq 1$ and Algorithm $1$ always sets $x_{1t} = 1$ and Algorithm $2$ always sets $x_{2t} = 1$. Any alternating strategy has a cost of $1$ at any time $t$. The optimal cost is $1/2^p$ at any time $t$ by setting $x_{1t} = x_{2t} = 1/2$. The competitive ratio is $2^p$. We can generalize this example for $n$ algorithms ($n$ types of variables $x_{1t}, x_{2t}, ..., x_{nt}$). The same computation gives the competitive ratio of $n^p$.

This example shows that the competitive ratio (of the algorithm that alternates between two solutions) is not captured by any function intrinsically depending only on $\lambda$ and $\mu$ (as there is a parameter $n$ in the competitive ratio). In this particular example, for a degree-$p$ polynomial with positive coefficients, $\lambda$ and $\mu$ are finite bounded values independent of $n$ (precise values can be found in Section~\ref{sec:energy}).

\clearpage

\section{Complete proof of Lemma~\ref{lem:bound-x} in \cref{sec:covering}} \label{apix:lemma-proof}
\setcounter{theorem}{0}
\begin{lemma}
	Let $e$ be an arbitrary resource.
	At any moment $\tau$ during the execution of the algorithm,
	when $t$ constraints have already been released, it always holds that
	$$
	x_{e}	\geq  \frac{\eta}{b^{t_{e}^{*}}_{e}(A) \ d}
			\left[ \exp\biggl( \frac{\ln(1+2d^{2}/\eta)}{\beta_{e}}
					\cdot \sum_{A: e \notin A} \sum_{t' \le t} b^{t'}_{e}(A) \cdot \alpha^{t'}_{A} \biggr) - 1 \right]
	$$
	where $b^{t_{e}^{*}}_{e}(A)$ is defined in the algorithm on line~\ref{algo-covering:bmax}.
\end{lemma}
\begin{proof}
	Let us fix a resource $e$ and prove the lemma by induction. At the beginning of the execution, when no constraint has been released yet, both sides of the lemma are 0.
	Let us assume that the lemma holds until the release of the $t^{\text{th}}$ constraint $\sum_{e} a^{t}_{e} x_{e} \geq 1$.
	Consider a moment $\tau$ during the algorithm's execution
	and let $A^{*}$ be the current set of resources $e'$ such that $x_{e'} = 1$.
	If at time~$\tau$, $x_{e} = 1$, then by the algorithm's design, the set $A^{*}$ has been updated such that
	$e \in A^{*}$. As such, the increasing rates of both sides in the lemma inequality are 0.
	In the remaining of the proof, we assume that  $x_{e} < 1$.
	We recall that by the algorithm's design, $\beta_{e} \geq \frac{1}{\lambda} \nabla_{e} F(\vect{x})$.
	We consider two cases $\beta_{e} > \frac{1}{\lambda} \nabla_{e} F(\vect{x})$
	and $\beta_{e} = \frac{1}{\lambda} \nabla_{e} F(\vect{x})$.

	\textbf{Case 1: $\beta_{e} > \frac{1}{\lambda} \nabla_{e} F(\vect{x})$.}
	In this case, by the algorithm's design, the value of $\beta_{e}$ remains unchanged at time~$\tau$ (line \ref{algo-covering:beta}) ($\frac{\partial \beta_{e}}{\partial \tau} = 0$).
	The lemma's right-hand side's derivative according to $\tau$ is
	\begin{align*}
	&\sum_{t' \le t} \frac{\partial \alpha^{t'}_{A^{*}}}{\partial \tau} \cdot
		\frac{b^{t'}_{e}(A^{*}) \ \eta }{b^{t_{e}^{*}}_{e}(A) \ d} \cdot \frac{\ln(1+2d^{2}/\eta)}{\beta_{e}}
			\cdot \exp\biggl( \frac{\ln(1+2d^{2}/\eta)}{\beta_{e} } \cdot \sum_{A: e \notin A} \sum_{t' \le t} b^{t'}_{e}(A)\ \alpha^{t'}_{A} \biggr) \\
	%
	&\leq \frac{\partial \alpha^{t}_{A^{*}}}{\partial \tau} \cdot
		\frac{b^{t}_{e}(A^{*}) \ \eta }{b^{t_{e}^{*}}_{e}(A) \ d} \cdot \frac{\ln(1+2d^{2}/\eta)}{\beta_{e}} \cdot \left( \frac{b^{t_{e}^{*}}_{e}(A) \ d}{\eta}\ x_{e} + 1 \right) \\
	%
	&= \frac{1}{\lambda \ln(1+2d^{2}/\eta)} \cdot
		\frac{b^{t}_{e}(A^{*}) \ \eta }{b^{t_{e}^{*}}_{e}(A) \ d} \cdot \frac{\ln(1+2d^{2}/\eta)}{\beta_{e}} \cdot \left( \frac{b^{t_{e}^{*}}_{e}(A) \ d}{\eta}\ x_{e} + 1 \right) \\
	%
	&\leq  \frac{b^{t}_{e}(A^{*}) \ x_{e}}{\lambda\ \beta_{e}} + \frac{\eta}{\lambda\ \beta_{e}\ d} \\
	%
	&\leq \frac{\partial x_{e}}{\partial \tau}
	\end{align*}
	%
	In the first inequality, we use the induction hypothesis and $\frac{\partial \alpha^{t}_{A^{*}}}{\partial \tau} > 0$
	and $\frac{\partial \alpha^{t'}_{A^{*}}}{\partial \tau} \leq 0$ for $t' < t$ and $\frac{\partial \beta_{e}}{\partial \tau} = 0$.
	The equality follows the increasing rate of $\alpha^{t}_{A^{*}}$.
	The last inequality is due to the increasing rate of $x_{e}$.
	The rate on the left-hand side is always larger than on the right-hand side, so the lemma inequality holds.

	\textbf{Case 2: $\beta_{e} = \frac{1}{\lambda} \nabla_{e} F(\vect{x})$.}
	In this case, by the algorithm's design, $\frac{1}{\lambda} \nabla_{e} F(\vect{x})$ is locally non-decreasing at $\tau$ (since otherwise,
	by line \ref{algo-covering:beta}, $\beta_{e}$ is not maintained to be equal to $\frac{1}{\lambda} \nabla_{e} F(\vect{x})$).
	Therefore, $\frac{\partial \beta_{e}}{\partial \tau} \geq 0$ and so $\partial \bigl(\frac{1}{\beta_{e}}\bigr)/\partial \tau \leq 0$.
	Hence, the derivative of the right-hand side of the lemma inequality according to $\tau$ is upper bounded by
	\begin{align*}
	\sum_{t' \le t} \frac{\partial \alpha^{t}_{A^{*}}}{\partial \tau} \cdot
		\frac{b^{t'}_{e}(A^{*}) \ \eta}{b^{t_{e}^{*}}_{e}(A) \ d} \cdot \frac{\ln(1+2d^{2}/\eta)}{\beta_{e}}
			\cdot \exp\biggl( \frac{\ln(1+2d^{2}/\eta)}{\beta_{e} } \cdot \sum_{A: e \notin A} \sum_{t' \le t} b^{t'}_{e}(A)\ \alpha^{t'}_{A} \biggr)
	\end{align*}
	which is bounded by $\frac{\partial x_{e}}{\partial \tau}$ by the same argument as the previous case. The lemma follows.
\end{proof}


\clearpage

\section{Complete proofs from application related
propositions} \label{sec:appix-proofs}
\setcounter{theorem}{0}

\subsection{Load Balancing}

\begin{proposition}
Algorithm~\ref{algo:covering} gives a
$O(\frac{1}{1 - \eta})$-consistent and $O\bigl((\log m) \log^{2} \frac{m}{\eta}\bigr)$-robust fractional solution
for the load balancing problem.
\end{proposition}
%
\begin{proof}
It is known that $\infty$-norm of a $m$-dim vector can be approximated by the $(\log m)$-norm,
in particular for $m \geq 2$,
$$
\|(\ell_{1}, \ell_{2}, \ldots, \ell_{m})\|_{\infty} \leq \|(\ell_{1}, \ell_{2}, \ldots, \ell_{m})\|_{\log m}
\leq m^{1/m} \|(\ell_{1}, \ell_{2}, \ldots, \ell_{m})\|_{\infty}
\leq 2 \|(\ell_{1}, \ell_{2}, \ldots, \ell_{m})\|_{\infty}.
$$
Hence, one can instead consider the objective of minimizing the  $(\log m)$-norm of the load vectors
while losing a constant factor of 2. More precisely, we consider the $(\log m)$-th power of the $(\log m)$-norm as the objective.
$$
\min \sum_{i=1}^{m} \biggl(\sum_{j} p_{ij} x_{ij}\biggr)^{\log m}
\qquad \text{s.t.} \qquad
\sum_{i=1}^{m} x_{ij} = 1 ~ \forall j
$$
%
The objective function is a polynomial of degree $\log m$. So its multilinear extension is \linebreak
$(O(k \ln(d/\eta))^{k-1}, \frac{k-1}{k \ln(1 + 2d^{2}/\eta)})$-locally smooth
with $k = \log m$ and $d = m$ (the maximal number of positive coefficients in a constraint).
Therefore, applying Theorem~\ref{thm:covering-formal}, the robustness (w.r.t the objective as  the $(\log m)$-th power of the $(\log m)$-norm)
is $O\bigl((\log m \log \frac{m}{\eta})^{\log m}\bigr)$.
Getting back to the $(\log m)$-norm objective by taking the $(\log m)$-root,
the robustness is  $O\bigl((\log m) \log^{2} \frac{m}{\eta}\bigr)$.
Hence, Algorithm~\ref{algo:covering} is $O(\frac{1}{1 - \eta})$-consistent and $O\bigl((\log m) \log^{2} \frac{m}{\eta}\bigr)$-robust.
\end{proof}

\subsection{Energy Minimization in Scheduling}

\begin{proposition}
Algorithm~\ref{algo:covering} gives a
$O(\frac{1}{1 - \eta})$-consistent and $O\bigl(k^{k} \log^{k} \frac{m}{\eta}\bigr)$-robust fractional solution
for the energy minimization problem.
\end{proposition}
%
\begin{proof}
The objective function $\sum_{i} \sum_{t} P(\sum_{j} s_{ij}(t))$ is a polynomial of degree $k = \max_{i} k_{i}$;
so its multilinear extension is
$(O(k \ln(m/\eta))^{k-1}, \frac{k-1}{k \ln(1 + 2m^{2}/\eta)})$-locally smooth
(the maximal number of positive coefficients in a constraint $d = m$).
Therefore, applying Theorem~\ref{thm:covering-formal},
Algorithm~\ref{algo:covering} provides a $O(\frac{1}{1 - \eta})$-consistent and $O\bigl(k^{k} \ln^{k} \frac{m}{\eta}\bigr)$-robust
fractional solution.
\end{proof}


\subsection{Online Submodular Mimimization}	\label{apix:sub-min}


\begin{proposition}
Algorithm~\ref{algo:covering} gives a
$O(\frac{1}{1 - \eta})$-consistent and $O\bigl( \frac{\log (d/\eta)}{1 - \kappa_{f}} \bigr)$-robust fractional  solution
for the submodular minimization under covering constraints.
\end{proposition}
\begin{proof}
Let $F$ be the multilinear extension of $f$.
It is sufficient to verify that $F$ is $\bigl(\frac{1}{1-\kappa_{f}},0\bigr)$-locally smooth.
Recall that, by definition of the multilinear extension,
$F(\vect{x}) = \mathbb{E} \bigl[ f(\one_{T})\bigr]$ where $T$ is a random set
such that a resource $e$ appears in $T$ with probability $x_{e}$. Moreover, as $F$ is linear in $x_{e}$, we have
%
\begin{align*}
\nabla_{e} F(\vect{x}) %= \frac{\partial F(\vect{x}) }{\partial x_{e}}
&= F(x_{1}, \ldots, x_{e-1}, 1, x_{e+1}, \ldots, x_{n}) - F(x_{1}, \ldots, x_{e-1}, 0, x_{e+1}, \ldots, x_{n}) \\
&= \mathbb{E} \biggl[ f\bigl(\one_{R \cup \{e\}}\bigr) - f\bigl(\one_{R}\bigr) \biggr]
\end{align*}
where $R$ is a random subset of resources $N \setminus \{e\}$ such that $e'$ is included with probability $x_{e'}$.
Therefore, to prove that $F$ is $(\lambda,\mu)$-locally-smooth, it is equivalent to show that,
for any set $S \subset \mathcal{E}$ and for any vectors $\vect{x}^{e} \in [0,1]^{n}$ for $e \in \mathcal{E}$,
%
\begin{equation*}	\label{eq:min-local-smooth-equiv}
\sum_{e \in S} \mathbb{E} \biggl[ f\bigl(\one_{R^{e} \cup \{e\}}\bigr) - f\bigl(\one_{R^{e}}\bigr) \biggr]
\leq \lambda f\bigl( \one_{S} \bigr) + \mu \mathbb{E} \biggl[ f\bigl(\one_{R}\bigr) \biggr]
\end{equation*}
%
where $R^{e}$ is a random subset of resources $N \setminus \{e\}$ such that $e'$ is included with probability $x^{e}_{e'}$
and $R$ is a random subset of resources $N \setminus \{e\}$ such that $e'$ is included with probability $\max_{e \in S} x^{e}_{e'}$.

The $\bigl(\frac{1}{1-\kappa_{f}},0\bigr)$-local smoothness of $F$ holds due to submodularity and Lemma~\ref{lem:curvature} (see below), so
for any subsets $R^{e}$, we have
\begin{align*}
	\sum_{e \in S} \left[ f\bigl(\one_{R^{e} \cup \{e\}}\bigr) - f\bigl(\one_{R^{e}}\bigr) \right]
		\leq \sum_{e \in S} \left[ f\bigl(\one_{\{e\}}\bigr) \right]
		\leq \frac{1}{1 -\kappa_{f}} \cdot f(\one_{S})
\end{align*}
Therefore, applying Theorem~\ref{thm:covering-formal}, the proposition follows.
%Algorithm~\ref{algo:covering} gives a fractional
%$O(\frac{1}{1 - \eta})$-consistent and $O\bigl( \frac{\log (d/\eta)}{1 - \kappa_{f}} \bigr)$-robust solution.
\end{proof}

\begin{lemma}		\label{lem:curvature}
	For any set $S$, it always holds that
	$$
	f(\one_{S}) \geq (1-\kappa_{f}) \sum_{e \in S} f(\one_{\{e\}}).
	$$
\end{lemma}
\begin{proof}
	Let $S = \{e_{1}, \ldots, e_{m}\}$ be an
	arbitrary subset of $\mathcal{E}$. Let $S_{i} = \{e_{1}, \ldots, e_{i}\}$ for $1 \leq i \leq m$ and $S_{0} = \emptyset$.
	We have
	\begin{align*}
	f(\one_{S})
	&\geq  f(\one_{\mathcal{E}}) -  f(\one_{\mathcal{E} \setminus S})
	= \sum_{i=0}^{m-1}  f(\one_{\mathcal{E} \setminus S_{i}}) - f(\one_{\mathcal{E} \setminus S_{i+1}})
	\geq \sum_{i=1}^{m}  f(\one_{\mathcal{E}}) - f(\one_{\mathcal{E} \setminus \{e_{i}\}}) \\
	&\geq (1 - \kappa_{f}) \sum_{i=1}^{m} f(\one_{e_{i}})
	\end{align*}
	where the first two inequalities are due to the submodularity of $f$, and the last inequality follows the definition of curvature.
\end{proof}

%!TEX root = ./main.tex

\section{Counter example for the performance of the algorithm of \cite{AnandGe22:Online-Algorithms}}
\label{sec:counter-example}


Anand, Ge, Kumar and Panigrahi \cite{AnandGe22:Online-Algorithms} recently proposed online algorithms for online covering problems with multiple expert solutions.
We show here a counter example that contradicts Theorem~$2.1$ presented in Section~$3$ of their paper.
In the proof of Theorem~$2.1$ the authors state that \textit{the total cost of the algorithm is at most $3$ times the potential $\phi$ at the beginning, i.e., at most $O(\log~K)$ times the \texttt{DYNAMIC} benchmark}. However, in our counter example the total cost of their algorithm is $O(L \log(K))$ times the \texttt{DYNAMIC} benchmark, where $L$ is an arbitrary large number.

\subsection{Setting}

Algorithm $1$ (from \cite{AnandGe22:Online-Algorithms}) receives solutions from $K$ experts. The authors denote with $x_i(j,s)$ the solution from expert $s$ for variable $i$ on constraint $j$. They assume that the expert solutions are tight, formally:
%
\[ \sum_{i=1}^{n} a_{ij}\ x_{i}(j, s) = 1 \ \ \ \ \forall\ s \in [K]\]
%
The algorithm's performance is compared to the \texttt{DYNAMIC} benchmark, which is the minimum cost solution that is supported by at least one expert at each step, formally:
%
\[\texttt{DYNAMIC} = \min_{\hat{\textbf{x}} \in \hat{X}} \sum_{i=1}^{n} c_i \hat{x}_i \textnormal{, where}\]
%
\[\hat{X} = \{\hat{\textbf{x}} : \forall\ i \in [n],\ \forall\ j \in [m],\ \exists\ s \in [K] \textnormal{ where the solution } x_i(j,s) \le \hat{x}_i \}\]
%
While a constraint is not satisfied, their algorithm updates each variable with an increasing rate of
%
\[\frac{dx_i}{dt} = \frac{a_{ij}}{c_i}(x_i + \delta_{ij})\]
%
where $\delta_{ij} = \frac{1}{K} \sum_{s=1}^{K} x_i(j,s)$ is the average of the experts' solutions for $x_i$ at the arrival of constraint $j$.
Algorithm 1 of \cite{AnandGe22:Online-Algorithms} scales down the problem with $0.5$, so it does not increase any variable above $0.5$ and satisfies each constraint with value $0.5$. The exact solution is obtained by doubling the variables at the end of the execution. (This descaling is an important aspect in the authors' proof.)

\subsection{Counter example}

In the following example we reveal in an online manner a linear program parametrized by $L$ with $K$ experts and observe the behavior of Algorithm $1$ (from \cite{AnandGe22:Online-Algorithms}). This example is an extension of the pathological input for the multiplicative weight update algorithm.

\medskip

\noindent \textbf{Objective}. The example has $(L \cdot K + 1)$ variables with uniform cost:
\[ \min\ x_1 + x_2 + \dots + x_{K} + \dots + x_{2K} + \dots + x_{LK} + x_{LK+1}\]

\noindent \textbf{Constraints}. There are $L$ batches of $(K - 1)$ constraints. The first constraint of each batch has $(K+1)$ variables. The last variable ($x_{LK+1}$) is present in every constraint in every batch, but none of the experts suggests to use this variable. Within a batch, each consecutive constraint has one less variable. The experts set each variable that appears in later batches to $0$. The first batch:
%
\begin{align*}
     & \ \ \ \ x_{1} + x_{2} + \dots + x_{(K-1)} + x_{K} + x_{LK+1} \ge 1\\
\textnormal{Expert}_{1}: & \hspace{0.5cm} 1 \hspace{0.6cm} 0 \hspace{0.42cm} \dots \hspace{0.53cm} 0  \hspace{1.23cm} 0 \hspace{0.8cm} 0 \\
\textnormal{Expert}_{2}: & \hspace{0.5cm} 0 \hspace{0.6cm} 1 \hspace{0.42cm} \dots \hspace{0.53cm} 0  \hspace{1.23cm} 0 \hspace{0.8cm} 0 \\
     \vdots  & \\
\textnormal{Expert}_{K-1}: & \hspace{0.5cm} 0 \hspace{0.6cm} 0 \hspace{0.42cm} \dots \hspace{0.53cm} 1  \hspace{1.23cm} 0 \hspace{0.8cm} 0 \\
\textnormal{Expert}_{K}: & \hspace{0.5cm} 0 \hspace{0.6cm} 0 \hspace{0.42cm} \dots \hspace{0.53cm} 0  \hspace{1.23cm} 1 \hspace{0.8cm} 0 \\
     & \hspace{1.15cm} x_{2} + \dots + x_{(K-1)} + x_{K} + x_{LK+1} \ge 1\\
\textnormal{Expert}_{1}: & \hspace{0.5cm} 1 \hspace{0.6cm} 1 \hspace{0.42cm} \dots \hspace{0.53cm} 0  \hspace{1.23cm} 0 \hspace{0.8cm} 0 \\
\textnormal{Expert}_{2}: & \hspace{0.5cm} 0 \hspace{0.6cm} 1 \hspace{0.42cm} \dots \hspace{0.53cm} 0  \hspace{1.23cm} 0 \hspace{0.8cm} 0 \\
     \vdots  & \\
\textnormal{Expert}_{K-1}: & \hspace{0.5cm} 0 \hspace{0.6cm} 0 \hspace{0.42cm} \dots \hspace{0.53cm} 1  \hspace{1.23cm} 0 \hspace{0.8cm} 0 \\
\textnormal{Expert}_{K}: & \hspace{0.5cm} 0 \hspace{0.6cm} 0 \hspace{0.42cm} \dots \hspace{0.53cm} 0  \hspace{1.23cm} 1 \hspace{0.8cm} 0 \\
     \vdots  & \\
     & \hspace{2.7cm} x_{(K-1)} + x_{K} + x_{LK+1} \ge 1\\
\textnormal{Expert}_{1}: & \hspace{0.5cm} 1 \hspace{0.6cm} 1 \hspace{0.42cm} \dots \hspace{0.53cm} 1  \hspace{1.23cm} 0 \hspace{0.8cm} 0 \\
\textnormal{Expert}_{2}: & \hspace{0.5cm} 0 \hspace{0.6cm} 1 \hspace{0.42cm} \dots \hspace{0.53cm} 1  \hspace{1.23cm} 0 \hspace{0.8cm} 0 \\
     \vdots  & \\
\textnormal{Expert}_{K-1}: & \hspace{0.5cm} 0 \hspace{0.6cm} 0 \hspace{0.42cm} \dots \hspace{0.53cm} 1  \hspace{1.23cm} 0 \hspace{0.8cm} 0 \\
\textnormal{Expert}_{K}: & \hspace{0.5cm} 0 \hspace{0.6cm} 0 \hspace{0.42cm} \dots \hspace{0.53cm} 0  \hspace{1.23cm} 1 \hspace{0.8cm} 0 \\
\end{align*}

During the first constraint of every batch, the experts' solutions form an identity matrix. With each disappearing variable in the consecutive constraints, experts who suggested to use variables which are no longer available, choose to set the variable with the smallest index. Consequently, $(K-1)$ experts suggest to use variable $x_{(K-1)}$ and one expert suggests to use $x_K$ during the last constraint in the first batch. The pattern of the experts' solutions are identical for each batch. The constraints of the $l^{th}$ batch ($ 1 \le l \le L)$ are:
%
\begin{align*}
     x_{(l-1) K + 1} + x_{(l-1) K + 2} + \dots + x_{(l-1) K + (K-1)} + x_{lK} + x_{LK+1} \ge & \ 1\\
     x_{(l-1) K + 2} + \dots + x_{(l-1) K + (K-1)} + x_{lK} + x_{LK+1} \ge & \ 1\\
     \vdots &\\
     x_{(l-1) K + (K-1)} + x_{lK} + x_{LK+1} \ge & \ 1\\
\end{align*}
%

\begin{claim}
The objective value of Algorithm $1$ (from \cite{AnandGe22:Online-Algorithms}) on our example is $O(L \log(K))$ times the \texttt{DYNAMIC} benchmark.
\end{claim}
%
\begin{proof}
The optimal solution $\vect{x}^{*}$ of the \texttt{DYNAMIC} benchmark is the solution in which $x^{*}_{LK+1} = 1$ and $x^{*}_{i} = 0$ for $i \neq LK + 1$.
We verify that $\vect{x}^{*} \in \hat{X}$. For each $i \neq l K$ where $1 \leq l \leq L$, and for each constraint $m$, $x^{*}_{i} \geq 0 = x_{i}(m,K)$.
For $i = l K$, and for each constraint $m$, $x^{*}_{i} \geq 0 = x_{i}(m,1) = x_{i}(m,2) = \ldots = x_{i}(m,K-1)$.
Moreover, $\vect{x}^{*}$ satisfies all constraints (since variable $x_{LK+1}$ appears in all constraints).
Hence, $\vect{x}^{*} \in \hat{X}$. Subsequently, the objective value of the \texttt{DYNAMIC} benchmark is $1$.

     By the design of Algorithm $1$, the increasing rate of $x_{LK+1}$ is zero throughout the execution, and the variables which are not part of the current constraint are not increased. During the first constraint of each batch, the increasing rate of the first $K$ variables in the batch is $1/K$, since the increasing rate of variable $x_i$ is $(x_i + \frac{1}{K} \sum_{s=1}^{K} x_i(1,s))$ and initially every variable is set to zero. At the second constraint, the increasing rate of the second variable in the batch is higher than the other variables' increasing rate, because the first expert also uses this variable in its solution. Therefore, the increasing rate of the second variable is $(x_{(l-1) K + 2} + 2/K)$, while the other remaining expert variables in the constraint have an increasing rate of $(x_i + 1/K)$. Following the same reasoning (apart from the first constraint in the batch), the variable with the smallest index in the constraint has a higher increasing rate, than the other variables. During the last constraint of each batch, the increasing rate of the last two remaining expert variables are
     $(x_{(l-1) K + (K-1)} + (K-1)/K)$ and $(x_{lK} + 1/K)$. Keeping the increasing rates and the constraint satisfaction in mind, we can lower bound the value of each variable:
     %
     \begin{align*}
          \frac{1}{K} \ \le& \ x_{(l-1) K + 1} \\
          \frac{1}{K-1} \ \le& \ x_{(l-1) K + 2} \\
          \frac{1}{K-2} \ \le& \ x_{(l-1) K + 3} \\
          & \ \vdots \\
          \frac{1}{3} \ \le& \ x_{(l-1) K + (K-2)} \\
          \frac{1}{2} \ \le& \ x_{(l-1) K + (K-1)} \\
          \frac{1}{K} \ \le& \ x_{lK} \\
     \end{align*}

     Summing the terms together, we get that the objective value increases at least with $O(\log K)$ during each batch. There are $L$ batches, so the total cost of Algorithm $1$ is at least $O(L \log(K))$,
     while the total cost of the \texttt{DYNAMIC} benchmark is $1$, which concludes the proof.
\end{proof}

\subsection{Comparison}

In this specific counter-example, the \texttt{LIN-COMB} benchmark is equivalent to the static best-expert benchmark, i.e., the solution of Expert$_K$. The objective value of \texttt{LIN-COMB} is $L$ (since the optimal solution sets $x_{lK}$ variables for $1 \leq l \leq L$  to 1 and other variables to 0). In this counter-example, the objective value of our algorithm is $O(L\log K)$. Consequently, our proposed algorithm is $O(\log K)$ competitive in the \texttt{LIN-COMB} benchmark.


\end{document}
