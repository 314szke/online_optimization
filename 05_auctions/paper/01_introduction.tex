\section{Introduction}	\label{sec:welfare}
%
In a general mechanism design setting, each player $i$ has a set of actions $\mathcal{A}_{i}$
for $1 \leq i \leq n$. Given an action $a_{i} \in \mathcal{A}_{i}$ chosen by each player $i$ for $1 \leq i \leq n$, which lead to the 
action profile $\vect{a} = (a_{1}, \ldots, a_{n}) \in \mathcal{A} = \mathcal{A}_{1} \times \ldots \times \mathcal{A}_{n}$, 
the auctioneer decides an outcome $o(\vect{a})$ among the set of feasible outcomes $\mathcal{O}$. 
Each player $i$ has a \emph{private valuation} (or \emph{type}) $v_{i}$ taking values in a parameter space $\mathcal{V}_{i}$.
For each outcome $o \in \mathcal{O}$, player $i$ has \emph{utility} $u_{i}(o,v_{i})$ depending on
the outcome of the game and its valuation $v_{i}$. 
Since the outcome $o(\vect{a})$ of the game is determined by the action profile $\vect{a}$, 
the utility of a player $i$ is denoted as $u_{i}(\vect{a};v_{i})$. 
We are interested in auctions that in general consist of an allocation rule and a payment rule.
Given an action profile $\vect{a} = (a_{1}, \ldots, a_{n})$, the auctioneer decides an allocation 
and a payment $p_{i}(\vect{a})$ for each player $i$. 
Then, the \emph{utility} of player $i$ with valuation $v_{i}$, following the quasi-linear utility model, 
is defined as $u_{i}(\vect{a};v_{i}) = v_{i} - p_{i}(\vect{a})$. 
The \emph{social welfare} of an auction is defined as the total utility of all participants (the players and the auctioneer):  
$\textsc{Sw}(\vect{a};\vect{v}) =  \sum_{i=1}^{n} u_{i}(\vect{a};v_{i}) + \sum_{i=1}^{n} p_{i}(\vect{a})$. 

In the paper, we consider incomplete-information settings. 
In contrast to the full-information settings where private valuations are deterministically determined, in 
incomplete-informations settings the valuation vectors $\vect{v}$ (in which each component is the valuation of a player) 
is drawn from a publicly known distribution $\vect{F}$ with density function $\vect{f}$. 
Let $\Delta(\mathcal{A}_{i})$ be the set of probability distributions over the actions 
in $\mathcal{A}_{i}$.  A strategy of a player is a mapping $\sigma_{i}: \mathcal{V}_{i} \rightarrow \Delta(\mathcal{A}_{i})$
from a valuation $v_{i} \in \mathcal{V}_{i}$ to a distribution over actions $\sigma_{i}(v_{i}) \in \Delta(\mathcal{A}_{i})$.

\begin{definition}[Bayes-Nash equilibrium]
A strategy profile $\vect{\sigma} = (\sigma_{1}, \ldots, \sigma_{n})$ is a 
\emph{Bayes-Nash equilibrium (BNE)}  if for every player $i$, for every valuation $v_{i} \in \mathcal{V}_{i}$, and 
for every action $a'_{i} \in \mathcal{A}_{i}$:
\begin{align*}
\E_{\vect{v}_{-i} \sim \vect{F}_{-i}(v_{i})} \left[ \E_{\vect{a} \sim \vect{\sigma}(\vect{v})} \left[ u_{i}(\vect{a};v_{i}) \right] \right]
\geq 
\E_{\vect{v}_{-i} \sim \vect{F}_{-i}(v_{i})} \left[ \E_{\vect{a}_{-i} \sim \vect{\sigma}_{-i}(\vect{v}_{-i})} \left[ u_{i}(a'_{i},\vect{a}_{-i};v_{i}) \right] \right]
\end{align*}
\end{definition}
%
For a vector $\vect{w}$, we use $\vect{w}_{-i}$ to denote the vector $\vect{w}$ with the $i$-th component removed. 
Besides, $\vect{F}_{-i}(v_{i})$ stands for the probability distribution over all players other than $i$
conditioned on the valuation $v_{i}$ of player $i$. 

The price of anarchy of Bayes-Nash equilibria of an auction is defined as 
$$
\inf_{\vect{F}, \vect{\sigma}} \frac{\E_{\vect{v} \sim \vect{F}}\bigl[ \E_{\vect{a} \sim \vect{\sigma}(\vect{v})}[\textsc{Sw}(\vect{a};\vect{v})] \bigr]}{\E_{\vect{v} \sim \vect{F}} \bigl[ \textsc{Opt}(\vect{v})\bigr]}
$$
where the infimum is taken over Bayes-Nash equilibria $\vect{\sigma}$ and 
$\textsc{Opt}(\vect{v})$ is the optimal welfare with valuation profile $\vect{v}$.

In the paper, we consider discrete settings of valuations and payments, i.e., 
there are only a finite (large) number of possible valuations
and payments. The main purpose of restricting to discrete settings is that we can use tools from linear programming.
The continuous settings can be done by considering successively finer discrete spaces.   


\subsection{Smooth Auctions}	\label{sec:smooth-auction}

In this section, we show that the primal-dual approach also captures the smoothness framework in 
studying the inefficiency of Bayes-Nash equilibria in incomplete-information settings. 
Smooth auctions have been defined by \citet{Roughgarden15:The-price-of-anarchy} 
and \citet{SyrgkanisTardos13:Composable-and-efficient}. The definitions are slightly different but both are inspired 
by the original smoothness argument \cite{Roughgarden15:Intrinsic-robustness} and 
all known smoothness-based proofs can be equivalently analyzed by one of these definitions.  
In this section, we consider the definition of smooth auctions in \cite{Roughgarden15:The-price-of-anarchy}
and revisit the price of anarchy bound of smooth auctions.
In the end of the section, we show that a similar proof carries through the smooth auctions defined by
\citet{SyrgkanisTardos13:Composable-and-efficient}.
  

\begin{definition}[\cite{Roughgarden15:The-price-of-anarchy}]	\label{def:smooth-auctions}
For parameters $\lambda, \mu \geq 0$, an auction is $(\lambda,\mu)$-\emph{smooth} if for every valuation profile 
$\vect{v} = (v_{1}, \ldots, v_{n})$,
there exists action distribution $D^{*}_{1}(\vect{v}), \ldots,D^{*}_{n}(\vect{v})$ over $\mathcal{A}_{1}, \ldots, \mathcal{A}_{n}$
such that, for every action profile $\vect{a}$,
%
\begin{align}	\label{eq:smooth-auctions}
\sum_{i} \E_{a^{*}_{i} \sim D^{*}_{i}(\vect{v})} \bigl[ u_{i}(a^{*}_{i},\vect{a}_{-i}; v_{i}) \bigr]
	\geq \lambda \cdot \textsc{Sw}(\vect{a}^{*};\vect{v}) - \mu \cdot \textsc{Sw}(\vect{a};\vect{v})  
\end{align}
\end{definition}
