\documentclass[11pt,a4paper]{article}
%\usepackage[pdftex]{graphicx,color}
\usepackage{graphicx,color}
\usepackage{amsmath,amssymb,dsfont,fullpage,epsfig,multirow,longtable}
\usepackage{epsfig,epstopdf,hhline}
\usepackage{cases}
\usepackage{wrapfig}
\usepackage{algorithm}
%\usepackage[noend]{algorithmic}
\usepackage{algorithmic}
\renewcommand{\algorithmicrequire}{\textbf{Input:}}
\renewcommand{\algorithmicensure}{\textbf{Output:}}
\usepackage[numbers]{natbib}
\usepackage[top=0.5 in, bottom=1 in, left=1 in, right=1 in, letterpaper]{geometry}
\usepackage{tikz}
\usetikzlibrary{arrows,positioning,decorations.pathreplacing,shapes}

\usepackage{thmtools}
\usepackage{thm-restate}
\usepackage{enumerate}

% Ours
\usepackage{enumerate}
\usepackage{mdframed}
\usepackage{setspace}
\usepackage{caption}
\setlength{\textfloatsep}{8pt}

%\declaretheorem[name=Theorem]{thm}

% for affiliation
\usepackage{authblk}

%%%%%%%%%% Start TeXmacs macrosƒ
\newenvironment{proof}{\noindent\emph{Proof\ }}{\hspace*{\fill}$\Box$\medskip}
\newenvironment{claimproof}{\noindent\emph{Proof of claim\ }}{\hspace*{\fill}$\Box$\medskip}
\newenvironment{plainproof}{\noindent\emph{Proof\ }}{}
\newtheorem{theorem}{Theorem}
\newtheorem{definition}{Definition}
\newtheorem{lemma}{Lemma}
\newtheorem{claim}{Claim}
\newtheorem{proposition}{Proposition}
\newtheorem{corollary}{Corollary}
\usepackage{amsmath}
\usepackage{paralist}
\usepackage{framed}

%%comment in algorithms
\renewcommand{\algorithmiccomment}[1]{\hfill  {\small  \tt \# #1}}

\newcommand\restr[2]{{% we make the whole thing an ordinary symbol
  \left.\kern-\nulldelimiterspace % automatically resize the bar with \right
  #1 % the function
  \vphantom{\big|} % pretend it's a little taller at normal size
  \right|_{#2} % this is the delimiter
  }}

\newcommand{\pred}{\texttt{pred}}
\newcommand{\vect}[1]{\ensuremath{\mathbf{#1}}}
\newcommand{\one}{\ensuremath{\mathds{1}}}
\newcommand{\PoA}{\text{PoA}}
\newcommand{\E}{\ensuremath{\mathbb{E}}}
% bullet dot
\newcommand\sbullet[1][.75]{\mathbin{\vcenter{\hbox{\scalebox{#1}{$\bullet$}}}}}

\usepackage[utf8]{inputenc} % Required for inputting international characters
\usepackage[T5]{fontenc} % Output font encoding for international characters

\usepackage{color, colortbl}
\usepackage{hyperref}
\hypersetup{colorlinks,
            linkcolor=blue,
            citecolor=blue,
            urlcolor=magenta,
            linktocpage,
            plainpages=false}
\usepackage[capitalise,noabbrev]{cleveref}



\newcommand{\comment}[1]{\textcolor{blue}{{\footnotesize
#1}}\marginpar{\raggedright\tiny \textcolor{blue}{Comment}}}


\begin{document}

\textbf{Problem:}

\begin{itemize}
    \item $n$ buyers
    \item each buyer $i$ with budget $B_i$
    \item evaluation function: $v_i : [0,1] -> \mathbb{R}^+$
    \begin{itemize}
        \item concave
        \item non-decreasing
    \end{itemize}
    \item bid: $0 \le b_i \le B_i$
    \item sum of bids: $\sum_i b_i = B$
    \item fraction received: $d_i = \frac{b_i}{B}$
    \item utility: $u_i(\vec{b}) = v_i(d_i) - b_i$
    \item effective welfare: $ew_i(\vec{b}) = \min\{B_i,\ v_i(d_i)\}$
\end{itemize}

\[
    \textnormal{Social welfare: } \max_{d} \left\{\sum_{i=1}^{n}  v_i(d_i) \right\} \qquad \textnormal{Effective welfare: } \max_{d} \left\{\sum_{i=1}^{n}  \min\{B_i,\ v_i(d_i)\} \right\}
\]

\vspace{1cm}
\textbf{Formulation:}
A drawback of this formulation is that we compare to only integral assignments. 

\begin{minipage}[t]{0.59\textwidth}
	\begin{align*}
		\max  \sum_{S \subseteq \mathcal{S}} &c_{S}\ z_{S} \\
		(\alpha) \hspace{1cm} \sum_{i=1}^{n} \ x_{i} &= 1 & &  \\
		(\beta_i) \hspace{0.82cm} \sum_{S: i \in S} z_{S}  &= x_{i}	& & \forall i \\
		(\gamma) \hspace{1cm} \sum_{S \subseteq \mathcal{S}} z_{S} &= 1 & & \\
		x_{i} &\in [0,1] & & \forall i\\
		z_{S} &\in \{0,1\} & & \forall S \subseteq \mathcal{S}\\
	\end{align*}
\end{minipage}
\begin{minipage}[t]{0.3\textwidth}
	\begin{align*}
		\min \alpha &+ \gamma \\
		\alpha &\ge \beta_i  &  \forall i \\
		\sum_{i \in S} \beta_{i} + \gamma &\ge c_{S}  & & \forall S \subseteq \mathcal{S}\\
		\alpha, \gamma &\geq 0 & & \\
		\beta_i &\geq 0 & & \forall i
\end{align*}
\end{minipage}


\vspace{1cm}
\textbf{Setting the dual variables:}

\begin{minipage}[t]{0.3\textwidth}
\begin{align*}
    \alpha &= \frac{\mu+1}{\mu} \max_i \left\{ u_i(\vec{b}) \right\} \\
    \gamma &= \mu \sum_{i} b_i \\
    \beta_i &= \frac{\mu+1}{\mu} u_i(\vec{b})
\end{align*}
\end{minipage}
\begin{minipage}[t]{0.6\textwidth}
\end{minipage}

\section{A stronger formulation}

\begin{minipage}[t]{0.59\textwidth}
	\begin{align*}
		&& \max  \sum_{S \subseteq \mathcal{S}} &c_{S}\ z_{S} \\
		(\alpha_{i}) && \sum_{d_{i}}  x_{i,d_{i}} &= 1 & & \forall i \\
		(\beta_{i,d_{i}}) && \sum_{S: (i,d_{i}) \in S } z_{S} &= x_{i,d_{i}} & & \\
		(\gamma) && \sum_{S} z_{S}  &= 1	& & \\
		&& x_{i,d_{i}}, z_{S} &\geq 0 & & \forall i, \forall S \subseteq \mathcal{S}\\
	\end{align*}
\end{minipage}
\begin{minipage}[t]{0.3\textwidth}
	\begin{align*}
		\min \sum_{i} \alpha_{i} &+ \gamma \\
		\alpha_{i} &\ge \beta_{i,d_{i}}  & &   \forall i,d_{i} \\
		\sum_{(i,d_{i}) \in S} \beta_{i,d_{i}} + \gamma &\geq c_{S}  & & \forall S \subseteq \mathcal{S}\\
\end{align*}
\end{minipage}
%
One should see $d_{i}$ as the fraction that player $i$ gets in an arbitrary assignment (not necessarily in an equilibrium). 
The first constraint is the sum over all possible fraction $d_{i}$: there should be a fraction $d_{i}$ corresponding to player $i$. 


For the proof of PoA at least 1/2, define $\alpha_{i}$ to be the utility of player $i$, $\beta_{i,d_{i}}$ is the utility of player $i$ if he plays 
the bid $\mu d_{i} \Gamma$ while the other players play their equilibrium bids and $\gamma = \Gamma = \sum_{i} b_{i}$. 

\section{Another formulation with better bound}

\begin{minipage}[t]{0.59\textwidth}
	\begin{align*}
		&& \max  \sum_{S \subseteq \mathcal{S}} &c_{S}\ z_{S} \\
		(\beta) && \sum_{i=1}^{n} \sum_{d_{i}} d_{i} \sum_{S: (i,d_{i}) \in S } z_{S} &= 1 & & \\
		(\gamma) && \sum_{S} z_{S}  &= 1	& & \\
		&& z_{S} &\geq 0 & & \forall i, \forall S \subseteq \mathcal{S}\\
	\end{align*}
\end{minipage}
\begin{minipage}[t]{0.3\textwidth}
	\begin{align*}
		\min \sum_{i} \beta &+ \gamma \\
		\sum_{(i,d_{i}) \in S} d_{i} \beta + \gamma &\geq c_{S}  & & \forall S \subseteq \mathcal{S}\\
\end{align*}
\end{minipage}

Let $d^{*}$ be a Nash equilibrium.  
To get the tight bound, define $\beta = \hat{U}'(d^{*}_{i})$ (the latter term is the same for every $i$), $\gamma = \sum_{i=1}^{n} \bigl( U(d^{*}_{i}) - d^{*}_{i}\hat{U}'(d^{*}_{i}) \bigr)$. 
The notation is taken in Chapter 21 of AGT book. 

\subsection{Optimal bound for effective welfare}
Consider an equilibrium $\vect{b}$. Let $d_{i}$ be the fraction received by $i$ in the equilibrium, i.e., $d^{*}_{i} = \frac{b_{i}}{\sum_{j=1}^{n} b_{j}}$.
Using the KKT conditions (similar to Chapter 21 of AGT book), for any bidder $i$ with the equilibrium bid $0 < b_{i} < B_{i}$, we have
$$
\hat{V}'_{i}\biggl( \frac{b_{i}}{\sum_{j=1}^{n} b_{j}} \biggr) 
= V'_{i}\biggl( \frac{b_{i}}{\sum_{j=1}^{n} b_{j}} \biggr) \cdot \biggl( 1 - \frac{b_{i}}{\sum_{j=1}^{n} b_{j}} \biggr) 
= \sum_{j=1}^{n} b_{j}
$$
Note that for all bidders $i$ and $i'$ such that $0 < b_{i}, b_{j} < B_{i}$ then 
$ \hat{V}'_{i}(d^{*}_{i})  = \hat{V}'_{i'}(d_{i'})$.

Define 
$$
\beta = \hat{V}'_{i}(d^{*}_{i}) 
$$
and
$\gamma = \sum_{j=1}^{n} \gamma_{i}$
where
$$
\gamma_{i} = \begin{cases} 
			B_{i} \text{ if } V_{i}(d^{*}_{i}) \geq B_{i}, \\
			2 V_{i}(d^{*}_{i}) -   d^{*}_{i} \hat{V}'_{i}(d^{*}_{i})
			\end{cases}
$$

\paragraph{Feasibility.}
The dual constraints reads
\begin{align*}
\sum_{(i,d_{i}) \in S} d_{i} \beta + \gamma &\geq c_{S} \\
%
\Leftrightarrow 
\sum_{(i,d_{i}) \in S} d_{i} \hat{V}'_{i}(d^{*}_{i})  + \sum_{i=1}^{n} \gamma_{i} &\geq \sum_{(i,d_{i}) \in S} \min\{V_{i}(d_{i}), B_{i}\} 
\end{align*}
%
We prove the above inequality for each term $i$. If $\gamma_{i} = B_{i}$ then it is trivial. Now assume that 
$\gamma_{i} = 2 V_{i}(d^{*}_{i}) -   d^{*}_{i} \hat{V}'_{i}(d^{*}_{i})$ (meaning that $V_{i}(d^{*}_{i}) < B_{i}$). 
We have
%
\begin{align*}
\hat{V}'_{i}(d^{*}_{i})  + \gamma_{i}  
&= d_{i} \hat{V}'_{i}(d^{*}_{i})  + 2 V_{i}(d^{*}_{i}) -   d^{*}_{i} \hat{V}'_{i}(d^{*}_{i}) 
= 2 V_{i}(d^{*}_{i}) + \hat{V}'_{i}(d^{*}_{i}) (d_{i} - d^{*}_{i}) \\
%
&= 2 V_{i}(d^{*}_{i}) + V'_{i}(d^{*}_{i})(1 - d^{*}_{i}) (d_{i} - d^{*}_{i}) \\
%
&= V_{i}(d^{*}_{i}) + V'_{i}(d^{*}_{i})(d_{i} - d^{*}_{i}) + V_{i}(d^{*}_{i})  - V'_{i}(d^{*}_{i}) d^{*}_{i} (d_{i} - d^{*}_{i}) \\
%
&\geq V_{i}(d_{i}) + V_{i}(d^{*}_{i})  - V'_{i}(d^{*}_{i}) d^{*}_{i} (d_{i} - d^{*}_{i}) \\
%
&\geq V_{i}(d_{i}) + V_{i}(d^{*}_{i})  - V'_{i}(d^{*}_{i}) \frac{d^{*}_{i}}{d_{i}} (d_{i} - d^{*}_{i})
\end{align*}
The first inequality is due to the concavity of $V_{i}$ and the second holds since $d_{i} \leq 1$.
It remains to prove that $V_{i}(d^{*}_{i})  - V'_{i}(d^{*}_{i}) \frac{d^{*}_{i}}{d_{i}} (d_{i} - d^{*}_{i}) \geq 0$. 
If $d_{i} \leq d^{*}_{i}$ then the inequality follows immediately. Assume that $d_{i} = \rho \cdot d^{*}_{i}$ for $\rho > 1$. 
Therefore, 
\begin{align*}
V_{i}(d^{*}_{i})  - V'_{i}(d^{*}_{i}) \frac{d^{*}_{i}}{d_{i}} (d_{i} - d^{*}_{i}) 
&= V_{i}(d^{*}_{i})  - \frac{\rho - 1}{\rho} V'_{i}(d^{*}_{i}) d^{*}_{i} \\
%
&\geq V_{i}(d^{*}_{i})  - V'_{i}(d^{*}_{i}) d^{*}_{i} \geq V_{i}(0) \geq 0
\end{align*}
The second inequality follows the concavity of $V_{i}$. We deduce that the feasibility holds. 

\paragraph{Primal and Dual.}
The ratio between primal and dual is at most 2. 

\paragraph{Remark.} To complete the analysis, one must consider cases where all equilibrium bids are either $B_{i}$ or 0. The PoA, in this case, is 1.  

\end{document}