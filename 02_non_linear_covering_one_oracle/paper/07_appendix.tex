%!TEX root = ./main.tex

\section*{Appendix}

\section{Missing example from the introduction} \label{apix:example-introduction}

The following example demonstrates why an approach where we alternate between the solutions of the prediction oracle and the primal-dual method does not work for non-linear objective functions.

Let us consider the objective of $x_{11}^p + ... + x_{1k}^p + x_{21}^p + ... + x_{2k}^p$ where $k$ and $p$ are some integers; and two algorithms: Algorithm $1$ and Algorithm $2$. At time $t$, the constraint is $x_{1t} + x_{2t} \geq 1$ and Algorithm $1$ always sets $x_{1t} = 1$ and Algorithm $2$ always sets $x_{2t} = 1$. Any alternating strategy has a cost of $1$ at any time $t$. The optimal cost is $1/2^p$ at any time $t$ by setting $x_{1t} = x_{2t} = 1/2$. The competitive ratio is $2^p$. We can generalize this example for $n$ algorithms ($n$ types of variables $x_{1t}, x_{2t}, ..., x_{nt}$). The same computation gives the competitive ratio of $n^p$. This ratio is not captured by any function intrinsically depending only on $\lambda$ and $\mu$ (as there is a parameter $n$ in the competitive ratio).


\section{Missing proofs from application related propositions} \label{sec:appix-proofs}
To apply Theorem \ref{thm:covering-formal} on specific problems, we need to determine the local-smoothness parameters for the multilinear extension.
\cite{Thang20:Online-Primal-Dual} provided these parameters for some broad classes of functions, in particular for polynomials with non-negative coefficients. Let $g_{\ell}: \mathbb{R} \rightarrow \mathbb{R}$ for $1 \leq \ell \leq L$
be degree-$k$ polynomials with non-negative coefficients and let $f:~\{0,1\}^{n}~\rightarrow~\mathbb{R}^{+}$ be the cost function
defined as $f(\one_{S}) = \sum_{\ell} b_{\ell} g_{\ell}\bigl( \sum_{e \in S} a_{e} \bigr)$ where $a_{e} \geq 0$ for every
$e$ and $b_{\ell} \geq 0$ for every $1 \leq \ell \leq L$.
Then the multilinear extension $F$ of $f$ is $(O(k \ln(d/\eta))^{k-1}, \frac{k-1}{k \ln(1 + 2d^{2}/\eta)})$-locally smooth.
We will use these parameters to derive the guarantees for the following problems.



%%% **************************
%%% **************************
%%% **************************

\subsection{Energy Minimization in Scheduling}

Reducing carbon emissions is a global effort in which energy-efficient algorithms play an essential role. For example, \cite{Albers10:Energy-efficient-algorithms} and \cite{GuCaiZengZhangJinDai:2019} studied energy-efficient algorithms for scheduling.

Given $m$ unrelated machines, we need to assign jobs that arrive online. Each job $j$ has a release date $r_{j}$, a deadline $d_{j}$, and a vector of machine dependent processing times $p_{ij}$. Contrary to performance-oriented scheduling, our goal is to design an assignment policy which can minimize the total energy consumption of the execution. To achieve this, we can adjust the machines' speed $s_{ij}(t)$ during the time interval $[t,t+1)$ for the execution of job $j$. Every machine $i$ has a non-decreasing energy power function $P_{i}(\cdot)$. Typically, $P_{i}(z) = z^{k_{i}}$ for some constant $k_{i} \geq 1$. The execution's total energy is $\sum_{i} \sum_{t} P(\sum_{j} s_{ij}(t))$.

In the classic online setting, this problem is well understood: there exists an $O(k^{k})$-competitive algorithm \cite{Thang20:Online-Primal-Dual} where $k = \max_{i} \{k_{i}\}$
and this bound is tight up to a constant factor \cite{Caragiannis08:Better-bounds}. In our extended study with predictions we represent this problem with the following non-linear program. The objective is $\min \sum_{i} \sum_{t} P(\sum_{j} s_{ij}(t))$ and the constraints are:
$$
\sum_{i=1}^{m} x_{ij} = 1,  \qquad \qquad \sum_{t = r_{j}}^{d_{j}-1} s_{ij}(t) \geq p_{ij} x_{ij}, \qquad  \qquad s_{ij}(t) \geq 0  \qquad \forall\ i,\ t
$$
where $x_{ij} \in \{0,1\}$ indicates whether job $j$ is assigned to machine $i$
and $s_{ij}(t) \geq 0$ denotes the speed of machine $i$ executing job $j$ during the time interval $[t, t+1)$.
The first constraint guarantees that job $j$ is assigned to some machine, and the second one ensures
that the job $j$ is completed on time (on the machine where the job is assigned). At the arrival of
job $j$, the prediction provides a solution $pred(x_{ij})$ and a speed $pred(s_{ij}(t))$ for $r_{j} \leq t \leq d_{j} - 1$.
Using our framework, we can deduce the following result.

\begin{proposition}
Algorithm~\ref{algo:covering} gives a
$O(\frac{1}{1 - \eta})$-consistent and $O\bigl(k^{k} \log^{k} \frac{m}{\eta}\bigr)$-robust fractional solution
for the energy minimization problem.
\end{proposition}
%
\begin{proof}
The objective function $\sum_{i} \sum_{t} P(\sum_{j} s_{ij}(t))$ is a polynomial of degree $k = \max_{i} k_{i}$;
so its multilinear extension is
$(O(k \ln(m/\eta))^{k-1}, \frac{k-1}{k \ln(1 + 2m^{2}/\eta)})$-locally smooth
(the maximal number of positive coefficients in a constraint $d = m$).
Therefore, applying Theorem~\ref{thm:covering-formal},
Algorithm~\ref{algo:covering} provides a $O(\frac{1}{1 - \eta})$-consistent and $O\bigl(k^{k} \ln^{k} \frac{m}{\eta}\bigr)$-robust
fractional solution.
\end{proof}



%%% **************************
%%% **************************
%%% **************************

\subsection{Online Submodular Mimimization}	\label{apix:sub-min}

Submodular minimization is a widespread subject in optimization and machine learning \cite{IwataFleischer01:A-combinatorial-strongly,Bachothers13:Learning-with,Bach16:Submodular-functions:,BalkanskiSinger:2020}. Let us consider the problem of minimizing an online monotone submodular function subject to covering constraints.
A set-function $f: 2^{\mathcal{E}} \rightarrow \mathbb{R}+$ is \emph{submodular} if
$f(S \cup e) - f(S) \geq f(T \cup e) - f(T)$ for all $S \subset T \subseteq \mathcal{E}$.
Let $F$ be the multilinear extension of a monotone submodular function $f$. Function $F$
admits two useful properties. First, if $f$ is monotone, then so is $F$. Second, $F$ is concave in
the positive direction, meaning that $\nabla F(\vect{x}) \geq \nabla F(\vect{y})$ for all $\vect{x} \leq \vect{y}$, where $\vect{x} \leq \vect{y}$ is defined as $x_{e} \leq y_{e} ~\forall e$.

To apply Algorithm~\ref{algo:covering}, we need to determine the local-smoothness parameters.
An important concept in studying submodular functions is the \emph{curvature}. Given a submodular
function $f$, the \emph{total curvature} $\kappa_{f}$ (\cite{ConfortiCornuejols84:Submodular-set-functions}) of $f$ is defined as
$
\kappa_{f} = 1 - \min_{e} \frac{f(\one_{\mathcal{E}}) - f(\one_{\mathcal{E} \setminus \{e\}})}{f(\one_{\{e\}})}.
$
Intuitively, the total curvature measures how far away $f$ is from being \emph{modular}. This concept of
curvature is used to determine both upper and lower bounds on the approximation ratios
for many submodular and learning problems (see \cite{ConfortiCornuejols84:Submodular-set-functions,GoemansHarvey09:Approximating-submodular,BalcanHarvey12:Learning-Submodular,Vondrak10:Submodularity-and-Curvature:,IyerJegelka13:Curvature-and-optimal,SviridenkoVondrak17:Optimal-approximation}).
The following lemma shows a useful property of the total curvature.

\setcounter{theorem}{7}
\begin{lemma}		\label{lem:curvature}
For any set $S$, it always holds that
$$
f(\one_{S}) \geq (1-\kappa_{f}) \sum_{e \in S} f(\one_{\{e\}}).
$$
\end{lemma}
\begin{proof}
Let $S = \{e_{1}, \ldots, e_{m}\}$ be an
arbitrary subset of $\mathcal{E}$. Let $S_{i} = \{e_{1}, \ldots, e_{i}\}$ for $1 \leq i \leq m$ and $S_{0} = \emptyset$.
We have
\begin{align*}
f(\one_{S})
&\geq  f(\one_{\mathcal{E}}) -  f(\one_{\mathcal{E} \setminus S})
= \sum_{i=0}^{m-1}  f(\one_{\mathcal{E} \setminus S_{i}}) - f(\one_{\mathcal{E} \setminus S_{i+1}})
\geq \sum_{i=1}^{m}  f(\one_{\mathcal{E}}) - f(\one_{\mathcal{E} \setminus \{e_{i}\}}) \\
&\geq (1 - \kappa_{f}) \sum_{i=1}^{m} f(\one_{e_{i}})
\end{align*}
where the first two inequalities are due to the submodularity of $f$, and the last inequality follows the definition of curvature.
\end{proof}

\setcounter{theorem}{6}
\begin{proposition}
Algorithm~\ref{algo:covering} gives a
$O(\frac{1}{1 - \eta})$-consistent and $O\bigl( \frac{\log (d/\eta)}{1 - \kappa_{f}} \bigr)$-robust fractional  solution
for the submodular minimization under covering constraints.
\end{proposition}
\begin{proof}
Let $F$ be the multilinear extension of $f$.
It is sufficient to verify that $F$ is $\bigl(\frac{1}{1-\kappa_{f}},0\bigr)$-locally smooth.
Recall that, by definition of the multilinear extension,
$F(\vect{x}) = \mathbb{E} \bigl[ f(\one_{T})\bigr]$ where $T$ is a random set
such that a resource $e$ appears in $T$ with probability $x_{e}$. Moreover, as $F$ is linear in $x_{e}$, we have
%
\begin{align*}
\nabla_{e} F(\vect{x}) %= \frac{\partial F(\vect{x}) }{\partial x_{e}}
&= F(x_{1}, \ldots, x_{e-1}, 1, x_{e+1}, \ldots, x_{n}) - F(x_{1}, \ldots, x_{e-1}, 0, x_{e+1}, \ldots, x_{n}) \\
&= \mathbb{E} \biggl[ f\bigl(\one_{R \cup \{e\}}\bigr) - f\bigl(\one_{R}\bigr) \biggr]
\end{align*}
where $R$ is a random subset of resources $N \setminus \{e\}$ such that $e'$ is included with probability $x_{e'}$.
Therefore, to prove that $F$ is $(\lambda,\mu)$-locally-smooth, it is equivalent to show that,
for any set $S \subset \mathcal{E}$ and for any vectors $\vect{x}^{e} \in [0,1]^{n}$ for $e \in \mathcal{E}$,
%
\begin{equation*}	\label{eq:min-local-smooth-equiv}
\sum_{e \in S} \mathbb{E} \biggl[ f\bigl(\one_{R^{e} \cup \{e\}}\bigr) - f\bigl(\one_{R^{e}}\bigr) \biggr]
\leq \lambda f\bigl( \one_{S} \bigr) + \mu \mathbb{E} \biggl[ f\bigl(\one_{R}\bigr) \biggr]
\end{equation*}
%
where $R^{e}$ is a random subset of resources $N \setminus \{e\}$ such that $e'$ is included with probability $x^{e}_{e'}$
and $R$ is a random subset of resources $N \setminus \{e\}$ such that $e'$ is included with probability $\max_{e \in S} x^{e}_{e'}$.

Indeed, the
$\bigl(\frac{1}{1-\kappa_{f}},0\bigr)$-local smoothness of $F$ holds due to the submodularity and Lemma~\ref{lem:curvature}:
for any subsets $R^{e}$, we have
\begin{align*}
	\sum_{e \in S} \left[ f\bigl(\one_{R^{e} \cup \{e\}}\bigr) - f\bigl(\one_{R^{e}}\bigr) \right]
		\leq \sum_{e \in S} \left[ f\bigl(\one_{\{e\}}\bigr) \right]
		\leq \frac{1}{1 -\kappa_{f}} \cdot f(\one_{S})
\end{align*}
Therefore, applying Theorem~\ref{thm:covering-formal}, the proposition follows.
%Algorithm~\ref{algo:covering} gives a fractional
%$O(\frac{1}{1 - \eta})$-consistent and $O\bigl( \frac{\log (d/\eta)}{1 - \kappa_{f}} \bigr)$-robust solution.
\end{proof}
