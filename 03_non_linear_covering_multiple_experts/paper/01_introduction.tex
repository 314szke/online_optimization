%!TEX root = ./main.tex

\section{Introduction}

%% short online algorithm
Online computation (coined by \cite{BorodinEl-Yaniv05:Online-computation}) is a well-established field in theoretical computer science. Online computational models consider inputs as request sequences, where each request arrives individually over time. After observing the current request, the problem-solving algorithm must perform an irrevocable action without additional information about the future. The performance of an online algorithm is typically measured by the competitive ratio metric, which is the worst ratio between the objective value obtained by the algorithm and that of the optimal solution. Intuitively, the competitive ratio measures the price of not knowing future requests. Our goal is to design performant algorithms within this metric.

%% machine learning as a way to go beyond the worst case
The traditional worst-case analysis is an indispensable framework in algorithm design and is central in the development of algorithms. Nevertheless, it can lead practical users to several pitfalls. Summarizing an algorithm's performance by a pathological worst-case can overestimate its performance on average. Many algorithms that perform well in practice admit mediocre theoretical guarantees, while others which are well-established in theory behave poorly, even on simple instances. Consequently, it is crucial to research theories that can better explain the performance of algorithms and advise algorithm design choices (\cite{Roughgarden19:Beyond-worst-case,Roughgarden20:Beyond-the-Worst-Case}).

%% recent trends
Much of the research focused on going beyond the worst-case paradigm is motivated by the spectacular advances of machine learning (ML). Specifically, ML methods can detect patterns among the arriving input requests and provide valuable insights for the online algorithms regarding future requests. \cite{LykourisVassilvtiskii18:Competitive-caching} introduced a general framework to integrate ML predictions into classical algorithm designs to surpass the worst-case performance limit.
Shortly after, \cite{MitzenmacherVassilvitskii20:Beyond-the-Worst-Case}
followed this line of research and studied online algorithms with predictions. As a result of these papers, many practically relevant online problems were revisited to enhance existing classical algorithms with ML predictions. For example, scheduling (\cite{LattanziLavastida20:Online-scheduling,Mitzenmacher20:Scheduling-with}), caching (\cite{LykourisVassilvtiskii18:Competitive-caching,Rohatgi20:Near-optimal-bounds,AntoniadisCoester20:Online-metric}), and ski rental (\cite{GollapudiPanigrahi19:Online-algorithms,KumarPurohit18:Improving-online}).

Even though predictions provide a glimpse of the future, there is no mathematical guarantee for their accuracy. Adjusting the algorithm's trust in the predictions is a significant challenge since online algorithms must make irrevocable decisions at each time step. Ideally, if the predictions are accurate, the algorithm should perform well compared to the offline setting. In contrast, if the predictions are misleading, the algorithm should maintain a competitive solution, similar to the online setting where no predictive information is available. In other words, online algorithms with predictions are expected to bring the best of both worlds: mathematical performance guarantees of classical algorithms and good future prediction capabilities of machine learning methods.

%% unified methods/primal-dual
To overcome the issue of unknown prediction accuracy, the authors of the works we cited previously exploited specific structures within the studied problems. \cite{BamasMaggiori20:The-Primal-Dual-method} presented a primal-dual method based technique to unify these different ad-hoc approaches and design online algorithms with predictions for various online problems. The primal-dual method is an elegant and powerful algorithm design technique (introduced by \cite{WilliamsonShmoys11:The-design-of-approximation}), especially for online algorithms (see \cite{BuchbinderNaor09:Online-primal-dual}). The work of \cite{BamasMaggiori20:The-Primal-Dual-method} focuses on problems with linear objectives and covering constraints. Until now, it remained an open question to design online algorithms with predictions for \emph{non-linear} covering problems. Non-linear objectives appear naturally in diverse application domains, such as energy and congestion management. Therefore, answering this open question has high theoretical interest and vital practical motivations. Our paper presents a framework to create online primal-dual algorithms with predictions for covering problems with non-linear objectives.
